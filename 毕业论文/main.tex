% -*- coding=utf-8 -*-
\documentclass[twoside, 12pt,openany, AutoFakeBold]{ctexbook}	% 文档类型
\usepackage{subfiles}  % 引入 subfiles 宏包
%%%%%%%%%%%%%%%%%%%%%%%%%%%%%%%%%%%%%%%%%%%%%%%%%%%%%%%%%%%%%%%%%%
%   使用说明:使用XeLatex(2019+)进行编译,建议关闭拼写检查
%	模版内容结构
%	- 格式相关:main.tex(学生无需处理操作)
%	- 论文内容:/latex目录下,包含封面(cover.tex)、中文摘要(cn_abstract.tex)、英文摘要(en_abstract)、正文(Chapter.tex)、谢辞(Thanks.tex)
%%%%%%%%%%%%%%%%%%%%%%%%%%%%%%%%%%%%%%%%%%%%%%%%%%%%%%%%%%%%%%%%%%

%=============================文档类型==============================
%	twoside命令:设置为双面排版,左右页边距会根据奇偶页自动调整
%	12pt:正文字体大小
%	openany:新的一章可以出现在左右两侧,方式ctexbook类型使得空白页出现
%	AutoFakeBold:实现粗体形式宋体
%
%=============================宏包调用==============================
\usepackage{wallpaper}	% 封面背景包
\usepackage{amsmath,mathtools,amsthm,amsfonts,amssymb,bm}	% AMS包
\usepackage{color}  % 字体背景颜色包
\usepackage{xeCJK}  % 中文字体配置包
\usepackage{ulem} % 下划线增强
\usepackage{setspace} % 行距控制
\usepackage{calc} % 计算长度
\usepackage{ulem}
\usepackage{setspace}
\usepackage{varwidth}  % 新增可变宽度盒子支持
\usepackage{xparse}    % 新增高级命令定义
\usepackage{soul}  % 更强大的下划线支持
\usepackage{tabularx}
\usepackage{setspace}
\usepackage{listings}
\usepackage{multicol} % 多列布局包

%==========================页面边距设置=============================
\usepackage[a4paper,left=25mm,right=25mm,top=25mm,bottom=25mm,headheight=15mm,headsep=4mm ,footskip=17.5mm]{geometry}	

%	页边距:上2.5cm,下2.5cm,左2.5cm,右2.5cm。
%	页眉顶端距离:1.5cm
%	页脚地段距离:1.75cm

%==========================页眉页脚设置=============================
\usepackage{fancyhdr}	% 页眉页脚包
\fancypagestyle{djtu}{
	\fancyhf{}		% 清空所有定义
	\fancyfoot[CE,CO]{\zihao{-5} \songti \thepage}      % 设置页脚为当前页码,居中,小五,数字1,2,3等
	\fancyhead[CE]{\zihao{-5} \heiti 大连交通大学 2026 届本科生综合毕业设计(论文)}  % 偶数页眉,居中,小五,黑体 
	\fancyhead[CO]{\zihao{-5} \heiti 大连交通大学 2024 届本科生综合毕业设计(论文)}  % 奇数页眉,居中,小五,黑体
	}
 \fancypagestyle{plain}{
 	\fancyhf{}		% 清空所有定义
 	\fancyfoot[CE,CO]{\zihao{-5} \songti \thepage}      % 设置页脚为当前页码,居中,小五,数字1,2,3等
 	\fancyhead[CE]{\zihao{-5} \heiti 大连交通大学 2026 届本科生综合毕业设计(论文)}  % 偶数页眉,居中,小五,黑体 
 	\fancyhead[CO]{\zihao{-5} \heiti 大连交通大学 2026 届本科生综合毕业设计(论文)}  % 奇数页眉,居中,小五,黑体
     }
% 定义摘要部分的页面样式:只显示页码,无页眉
\fancypagestyle{abstract}{
	\fancyhf{}		% 清空所有定义
	\fancyfoot[CE,CO]{\zihao{-5} \thepage}      % 设置页脚为当前页码,居中,小五
	}

%==========================字体配置设置=============================
\usepackage{fontspec}	% 字体配置宏包
\setmainfont{Times New Roman} % 设置英文字体为新罗马字体

%==========================段落相关设置=============================
\usepackage{setspace} 	% 段落行距包
\usepackage{indentfirst} %段落首行缩进命令包
\usepackage{lipsum}
\setlength{\baselineskip}{20pt}



% 行距的设定为20磅,需在\begin{document}中补充设置
\raggedbottom			% 防止latex排版导致段落间距变大
\setlength{\parindent}{2em}		% 首航缩进距离设置为2em,表示两个字符

%==========================章节格式设置=============================
\usepackage{ctex}
\ctexset{ % 用来定制正文里的标题格式
	chapter = {%
		name = {第,章},
		number = \chinese{chapter},              % 用中文数字显示章节号
		format = {\heiti \zihao{3} \centering},       % 设置章节标题为3号黑体且居中
		beforeskip = 0pt,			        % 段前行距0行	
		afterskip = 20pt,			        % 段后行距0行
		fixskip = true				        % 设置固定间距为true,抑制标题前后的多余间距
	},
	section = {%
		number = \normalfont\arabic{chapter}.\normalfont\arabic{section} \hspace{-1em}, % 编号格式为1.1,编号与题目名之间的空格
		format = {\heiti\zihao{4}\raggedright},   % section格式添加一条:左对齐
        beforeskip = 0pt,	% 段前行距0行
        afterskip = 0pt,	% 段后行距0行
	},
	subsection = {%
		number = \normalfont\arabic{chapter}.\normalfont\arabic{section}.\normalfont\arabic{subsection} \hspace{-1em},% 编号格式为1.1.1,编号与题目名之间的空格
		format = {\heiti\zihao{4}\raggedright},    % subsection格式添加一条:左对齐
        beforeskip = 0pt,	% 段前行距0行
        afterskip = 0pt		% 段后行距0行
	},
}

%==========================目录格式设置=============================
\usepackage{titletoc}		                % 目录定制包
\renewcommand{\contentsname}{目\quad 录}		% “目录”两字之间空X格

\titlecontents{chapter}
[0em] % 缩进量
{\bfseries\songti\zihao{-4}} % 控制字体字号
{\thecontentslabel\hspace{0em}} % 控制章节编号和名称的距离 
{} 
{\titlerule*[3pt]{.}\hspace{-1em}\contentspage} % 控制引导点密度,这里引入\hspace{-1em}的功能是防止页码和引导点之间有空格

\titlecontents{section}
[1em] % 缩进量
{\songti\zihao{-4}} % 控制字体字号
{\thecontentslabel\hspace{1em}} % 控制小节编号和名称的距离 
{} 
{\titlerule*[3pt]{.}\hspace{-1em}\contentspage} % 控制引导点密度,这里引入\hspace{-1em}的功能是防止页码和引导点之间有空格


\titlecontents{subsection}
[2em] % 缩进量
{\songti\zihao{-4}} % 控制字体字号
{\thecontentslabel\hspace{1em}} % 控制子节编号和名称的距离
{} 
{\titlerule*[3pt]{.}\hspace{-1em}\contentspage}% 控制引导点密度,这里引入\hspace{-1em}的功能是防止页码和引导点之间有空格

%==========================参考文献设置=============================
\usepackage[%
	backend=biber,				% 设置使用biber进行编译,也可以使用bibtex,但是功能更少
	style=gb7714-2015,			% 设置风格样式为国家标准gb7714-2015
	sorting=none					% 设置排序按照年份,名字,标题进行排序,若想按照引用顺序排序,将其设置为none即可
	]{biblatex}       			% 参考文献包
\addbibresource{bib/ref.bib}    % 加载参考文献的文件
\usepackage[hidelinks]{hyperref}  % 加载hyperref,实现参考文献引文跳转


%==========================图表环境设置=============================
\usepackage[inline]{enumitem}					% 列表工具包
\usepackage{graphicx,ragged2e}					% 插图工具包
\usepackage{subcaption}							% 子图标题包
\usepackage{bicaption}							% 图片标题包
\setlist{%	设置列表样式
	topsep=0.3em, 			% 列表顶端的垂直空白
	partopsep=0pt, 			% 列表环境前面紧接着一个空白行时其顶端的额外垂直空白
	itemsep=0ex plus 0.1ex, % 列表项之间的额外垂直空白
	parsep=0pt, 			% 列表项内的段落之间的垂直空白
	leftmargin=1.5em, 		% 环境的左边界和列表之间的水平距离
	rightmargin=0em, 		% 环境的右边界和列表之间的水平距离
	labelsep=0.5em, 		% 包含标签的盒子与列表项的第一行文本之间的间隔
	labelwidth=2em 			% 包含标签的盒子的正常宽度;若实际宽度更宽,则使用实际宽度。
}
\graphicspath{figure/}		% 设置图片存放目录
\usepackage{longtable,booktabs} % 跨页长表格
\usepackage{caption}
\captionsetup[table]{skip=0pt} % 设置表格标题与内容之间无间距
\captionsetup[figure]{skip=10pt} % 设置图片标题与内容之间无间距
\renewcommand{\arraystretch}{1.5} % 扩大表格每行文字距离
\DeclareCaptionLabelSeparator{quadspace}{\quad}  % 定义分隔符为\quad
\captionsetup[table]{
    labelsep=quadspace,                   % 标签与标题之间的分隔符
    justification=centering,              % 标题居中
    font=small,                           % 字体大小
}
\captionsetup[figure]{
    labelsep=quadspace,                   % 标签与标题之间的分隔符
    justification=centering,              % 标题居中
    font=small,                           % 字体大小
}
% 修改表格编号格式为 "章号-表格序号"
\renewcommand{\thetable}{\arabic{chapter}-\arabic{table}}
% 修改图片编号格式为 "章号-图片序号"
\renewcommand{\thefigure}{\arabic{chapter}-\arabic{figure}}

\usepackage{enumitem} % 必需宏包(用于(1)、(2)...)
\usepackage{pifont}   % 提供带圈数字符号

\newlist{customenum}{enumerate}{1}
\setlist[customenum]{
    label=(\arabic*),   % 标签格式
    leftmargin=*,       % 左对齐
    labelwidth=1em,     % 标签宽度
    align=left          % 对齐方式
}

\newlist{customsubenum}{enumerate}{1}
\setlist[customsubenum]{
    label=\ding{172},   % 使用CJK带圈数字符号
    leftmargin=2em,     % 调整缩进量
    labelwidth=1.5em,
    align=left
}

% 禁用mainmatter的分页,删除目录后一页产生的空白页
\makeatletter
\renewcommand{\mainmatter}{%
  \clearpage
  \@mainmattertrue
  \pagenumbering{arabic}
}
\makeatother



   % 导入公共导言(可选)

%==========================部分调试编译区=============================

% \includeonly{latex/cover}	% 只编译filelist.tex文件,调试时使用。		%其中“latex/”的作用是告诉LaTeX去子目录 latex/ 里找 cover.tex 文件



%==========================正文开始=============================

\begin{document}

%==========================格式补充设置=============================
\setlength{\baselineskip}{20pt}	% 设置行距为20磅

%==========================模版载入设置=============================
\frontmatter		% 关闭章节序号,页码默认使用小写罗马数字
\thispagestyle{empty}

%% -------------------- 封面 --------------------  
%======================个人信息录入=================================




% 题目:综合毕业设计题目
\newcommand\thesisTitle{XXX系统的分析与实现} 
   			
% 姓名:学生姓名
\newcommand\thesisAuthor{李春源} 	

% 学号:学生学号
\newcommand\thesisAuthorNum{221601021} 
		   
% 专业班级
\newcommand\thesisClass{R信息管理2021-X}	
	
% 学生所在院系(默认无需更改)		
\newcommand\thesisCollege{经济管理学院管理科学与工程系}				

% 指导教师及职称
% 第一组:
%	- 教师:李修飞、高贺、王德广、申广忠
%	- 职称:副教授、讲师、副教授、讲师
% 第二组
%	- 教师:王晗、莫东艳、杨晶、王春爽
%	- 职称:副教授、副教授、讲师、讲师
\newcommand\supervisor{王晗、莫东艳、杨晶、王春爽} 						% 指导教师
\newcommand\supervisortitle{副教授、副教授、讲师、讲师} 					% 指导教师职称

% 教师所在单位(默认无需更改)
\newcommand\instituteone{经济管理学院管理科学与工程系} 				 
\newcommand\institutetwo{轨道智能工程学院大数据科学与技术系} 			

% 完成时间
\newcommand\thesisDate{2025年11月16日}

%===================以下内容不要修改===========================

% 定义校徽图片路径:figure/logo.png
% \newcommand{\schoolLogo}{\includegraphics[width=1.02in,height=0.98in]{figure/logo.png}} % 校徽

% 定义校名图片路径:figure/djtu.jpg
\newcommand{\schoolName}{\includegraphics[width=8.56cm,height=1.98cm]{figure/djtu.jpg}} % 校名


% 以下为封面内容
{
    % 设置封面主要字体为黑体
	\heiti		% 设置封面主要字体为黑体
    \setlength\parindent{0em}   % 首行缩进设置为0
   
    % 设置校徽、校名、题目居中对齐,使用\begin{center}和\end{center}包裹                            
    \begin{center}
        \zihao{-3} 
        
    % 校徽和校名图片后间隔距离
%  \schoolLogo\\[0.5cm]
   \schoolName\\[1.0\baselineskip]

   % 毕业设计主标题
   {\songti \zihao{0} \bfseries 综合毕业设计} \\[1.0\baselineskip]
%   \vspace*{1.5\baselineskip}

   {\songti \zihao{-0} \bfseries (论文)}
   \vspace*{2.0\baselineskip}


   % 题目位置
   {\heiti \zihao{3} 题 \quad \quad 目\underline{\makebox[12cm][c]{\zihao{4} \thesisTitle}}}   
   \vspace*{2.5\baselineskip}
       
    \end{center}


    \renewcommand{\baselinestretch}{2}\selectfont                           % 设置声明的行间距
    {
        \begin{center}
    
    % 第一行:学生姓名 & 学号      
    {\zihao{4} 学生姓名}   \underline{\makebox[12em][c]{\zihao{4}\thesisAuthor}}{\zihao{4} 学\;\;\;\;\;\;\;号}   \underline{\makebox[12em][c]{\zihao{4}\thesisAuthorNum}}       
    \vspace*{0.5\baselineskip}  
            
    % 第二行:专业班级             
    {\zihao{4} 专业班级}   \underline{\makebox[29.25em][c]{\zihao{4}\thesisClass}}
    \vspace*{0.5\baselineskip}            
            
            
    % 第二行:所在院系      
    {\zihao{4} 所在院系}   \underline{\makebox[29.25em][c]{\zihao{4}\thesisCollege}}
    \vspace*{0.5\baselineskip}  
            
    % 第三行:指导教师      
    {\zihao{4} 指导教师}   \underline{\makebox[29.25em][c]{\zihao{4}\supervisor}}
    \vspace*{0.5\baselineskip}  
    
    % 第四行:职称      
    {\zihao{4} 职\;\;\;\;\;\;\;称}   \underline{\makebox[29.25em][c]{\zihao{4}\supervisortitle}}
    \vspace*{0.5\baselineskip}  
            
     % 第五行:所在单位1      
    {\zihao{4} 所在单位}   \underline{\makebox[29.25em][c]{\zihao{4}\instituteone}}
    \vspace*{0.5\baselineskip}  
	
	% 第六行:所在单位2      
    {\zihao{4} \quad \quad \quad \quad}   \underline{\makebox[29.25em][c]{\zihao{4}\institutetwo}}
	\par           
            

        \end{center}
    }
    
    \vspace{1.2cm}
    
    % 完成日期
    \begin{center}
 	{\zihao{4} 完成日期}
        \makebox[5cm][c]{\zihao{4} \thesisDate}        
    \end{center}
}












					% 导入封面

%% -------------------- 中英文摘要 -------------------- 
\fancypagestyle{plain}{
	\fancyhf{}		% 清空所有定义
	\fancyfoot[CE,CO]{\zihao{-5} \songti \thepage}      % 设置页脚为当前页码,居中,小五
	\renewcommand{\headrulewidth}{0pt}  % 去除页眉横线
	\renewcommand{\footrulewidth}{0pt}  % 去除页脚横线
	}

\pagestyle{abstract}
\setcounter{page}{1}					% 重制页面编号为1
\pagenumbering{Roman} 					% 摘要页码使用大写罗马数字
\chapter*{摘\quad  要}
\addcontentsline{toc}{chapter}{摘\quad  要}  % 添加到目录
\newcommand{\keywords}[1]{\text{\heiti #1}}
\vspace*{0.5\baselineskip}

现如今,教务系统在各大高校得到广泛地应用,这些系统提高了课程排课效率,也 为师生提供了许多优质的服务。但教务系统带来诸多便利的同时仍然存在着一些问题: 目前从学校教务管理系统中导出的师生课表,内容繁杂可读性较差,为生成简明扼要的 教师、学生(班级)课表,开展统计、分析、可视化等工作消耗大量的人力和时间。

本文通过调查研究近年来课表生成等技术的相关文献及实际案例,对现有问题进行 多方面的分析进而设计出系统的业务需求。系统采用UML建模方法进行分析,使用用例 图、时序图、类图等描述系统各个层面的设计,并以UML图作为后续开发的重要依据。 系统以桌 面 应用程 序 实现, 选 择Visual Studio 2019作 为主 要 开 发工 具, 采用 了 WPF+MVVM框架进行开发。 论文中包含了核心功能代码的详细描述。在实现过程中通 过白盒测试确保每一个逻辑通路的正确可用性。最后对系统整体进行测试,介绍了测试 环境以及系统测试结果。并对系统未来进行展望。

师生课表自动生成系统通过转换和整理原始课表数据,生成简明扼要的班级、教 师、教室课表,并提供批量查询、统计、分析、可视化等功能。帮助教师和教务人员更 轻松地查看和处理课表信息,方便教师开展教学工作、教务人员进行管理工作。


\vspace*{1.0\baselineskip}
\noindent
{\fontsize{10.5pt}{12pt}\selectfont \keywords{关键词:} 教务系统;课表生成;WPF;UML}% 关键词:} 之后填写中文关键词

\clearpage						% 跳到目录下一页

				% 导入中文摘要  
\chapter*{\bfseries ABSTRACT}
\addcontentsline{toc}{chapter}{ABSTRACT}  % 添加到目录
\vspace*{0.5\baselineskip}

The educational administration system has been widely applied in major universities, which has improved the efficiency of course scheduling and provided many high-quality services for teachers and students. However, while the educational administration system brings many conveniences, there are still some problems: the course schedules exported from the school's educational administration system are complex and have poor readability, and generating concise and concise teacher and student (class) schedules, conducting statistical analysis, visualization, and other work consumes a lot of manpower and time.

This paper investigates and studies the relevant literature and practical cases of course schedule generation technology in recent years, analyzes the existing problems from multiple perspectives, and designs the system's business requirements. The system uses the UML modeling method for analysis, and uses use case diagrams, sequence diagrams, class diagrams, etc. to describe the design of various levels of the system, and uses the UML diagrams as an important basis for subsequent development. The system is implemented as a desktop application, and Visual Studio 2019 is chosen as the main development tool, using the WPF+MVVM framework for development. The paper includes a detailed description of the core functional code. During the implementation process, white-box testing is used to ensure the correctness and availability of each logical path. Finally, the overall system is tested, the test environment and system test results are introduced, and the future of the system is described.

The automatic generation system of teacher and student course schedules converts and organizes the original course schedule data to generate concise class, teacher, and classroom schedules, and provides batch query, statistics, analysis, visualization and other functions. It helps teachers and educational administrators to view and process course schedule information more easily, facilitating teachers' teaching work and educational administrators' management work.


\vspace*{1.0\baselineskip}
\noindent
{\fontsize{10.5pt}{12pt}\selectfont \textbf{Key words:}academic affairs system; timetable generation; WPF;UML} % Keywords:} 之后填写英文关键词




\clearpage									% 跳到目录下一页

				% 导入英文摘要
 
%% -------------------- 目录 --------------------
\fancypagestyle{plain}{%
    \fancyhf{}  % 清空所有页眉页脚
    \renewcommand{\headrulewidth}{0pt}  % 去除页眉横线
    \renewcommand{\footrulewidth}{0pt}  % 去除页脚横线
} 
\pagestyle{empty}
\tableofcontents   						% 载入目录
\thispagestyle{empty}
\clearpage  % 确保mainmatter从当前页开始

%% -------------------- 页眉页脚 --------------------  
\mainmatter								% 开启章节序号计数,重置页码,页码使用阿拉伯数字

\fancypagestyle{plain}{\pagestyle{djtu}} % 设置默认的页面类型plain为自定义样式fancy
\pagestyle{djtu}
						% 设置页面布局为自定义的myfancy
						
%% -------------------- 正文 -------------------- 				
\chapter{绪\quad 论}

\section{系统开发背景}

现如今,各式教务系统在各大高校得到了越来越广泛地应用,这些系统极大地提高 了课程排课的准确度和效率,为学校工作人员节省了大量宝贵的时间。通过系统操作, 我们可以轻松地得到合理的排课安排,并且系统能够将大量的课程信息进行有效地整合, 大大减少了课程冲突和效率低下的问题。然而,尽管教务系统带来了诸多便利,但也存 在一些问题。例如,导出的课程表格内容往往十分繁杂,不易于阅读和理解表格中的数 据。此外,将教师课表和学生课表拆分出来的工作量大、效率低,不利于后续根据表格 对教师以及学生信息进行统计分析。为了解决这些问题,需要开发一款能够将教务表格 导入并自动生成简明美观的师生课表的系统。这样的系统将使教务工作更加高效、准确, 同时也为教师和学生提供了更方便的查询和统计分析功能。通过使用简明扼要的课表, 我们可以快速了解每个教师和学生的课程安排情况,及时发现并解决潜在的问题。此外, 此类系统还可以根据用户的个性化需求,定制不同的课表格式和展示方式,以满足用户 的需求。这类系统在应用环境中一般被称为教学辅助系统。

教学辅助软件在现代教育中发挥着越来越重要的作用。尤其是课表转换和分析功能, 为学校、教师和学生提供了许多便利和优势。

课表转换是指将原始的课程安排数据转换为更直观和易于理解的形式。通过教学辅 助软件,将教务系统输出的、繁琐的课程表单转换为直观的、易于阅读的课表格式。这 极大地提高了教师和学生查看和理解课程安排的效率。根据教师、教室和班级等信息自 动化生成课表,从而减少手动制作课表的错误和繁琐过程。

课表分析是指对课程安排数据进行深入分析,获取有关教学活动的信息并进行处理。 校内教务人员可以通过教学辅助软件的课表分析功能,更好地了解到教师的授课情况, 合理安排教师资源,确保教学质量。同时帮助教师明确各学期教学任务、授课时间等, 为教师教学提供依据。

\section{系统开发意义}

推动教务系统的发展,实现课表可视化、便捷化,能帮助教师更好地进行教学管理。 本课题根据大连交通大学教务系统输出的课程表单的实际情况,以及校内教师们的需求, 研究解决繁杂课表分解、师生课表可视化等问题的方法,并提供统计分析数据等数据处 理问题的解决方案。在此基础上,将问题转化为有条理的、符合程序设计流程的用户需 求,并产出清晰的数据流程。从而使教务系统课表更加简明清楚、易于查找,将教务管 理者从复杂的教务管理业务中解脱出来,为智慧校园建设添砖加瓦,也为学校教师提供 更好的服务与管理。

设计一款健壮性高、稳定性高、抗干扰性强、分析能力强的数据处理系统。通过使 用本系统,能够将教务系统中输出的课表进行分解整理,成为教师课表、班级课表和教室课表。通过查询功能,能够指定教师名称、课程名称或班级名称等关键词,查询其对 应的课程安排以及教室分配。通过统计分析功能,了解不同的教师在本学年或学期的教 学时间分配情况,班级分布情况,教学地点分配情况等相关的数据分析结果。也可以进 行关键词筛选,进行多重查询,了解在同一时间段,不同教师的课程安排以及教室情况 等等。为教师提供便利,解决教学生活中遇到的难题。

\section{可行性分析}

目前课表自动生成系统的相关研究较为丰富,研究成果较多,本课题的研究目的主 要是为大连交通大学教务系统产出的学年或学期课程表进行分解,成为个性化、易于查 看、简单美观的教师课表、班级课表和教室课表。并且辅助学校教师教学,增加统计分 析功能,让数据更加直观、可视化。在完成上述功能的同时尽可能保证系统安全稳定。

\subsection{技术可行性}

计划用 Visual Studio 2019 对系统主体进行开发。使用 Visual Studio 2019 具有很多 优势,主要包括以下几点:

(1)统一的集成开发环境:Visual Studio 2019 提供了一个统一的开发环境,能够 在一个地方处理多种编程语言和项目。不需要在不同的工具和编辑器之间切换,从而节 省了时间和精力。

(2)强大的调试功能:Visual Studio 2019 内置了先进的调试功能,快速定位和修 复代码中的错误。在制作系统时能够设置断点、监视变量的值、查看堆栈跟踪等,这些 工具使得调试过程更加高效和准确。

(3)丰富的工具和库:Visual Studio 2019 提供大量的工具和库,用于简化开发过 程、丰富应用程序的功能。运用代码自动完成、重构工具等各种扩展和插件,可以更快 地编写高质量的代码。

(4)强大的图形界面设计器:对于界面开发, Visual Studio 2019 的图形界面设计 器将会是一个强大的工具。它提供了直观的界面编辑器,能够将代码直接转化为图形, 并且可以通过改变图形来填充代码,加快界面开发的速度。

(5)开发者社区和资源库:Visual Studio 2019 拥有一个庞大的开发者社区和丰富 的资源库。在系统开发过程遇到困难时,能够尽快从其他开发人员的经验中获取支持、 经验、问题解决办法。

\subsection{操作可行性}

本系统操作界面设计相对简单,分为前台和后台两个主界面: 

前台界面是使用了 Visual Studio 特有的技术,旨在实现人机的交互。这意味着该系 统需要提供一个用户友好的界面,使用户能够轻松地与系统进行交互。通过使用 Visual Studio 的技术,该系统能够实现快速响应和高效的 PC 应用程序。用户可以通过简单的 页面操作,根据自己的需求选择相应的服务或功能。这种设计使得系统易于使用,即使 是对于没有深入计算机知识的人来说,也能够轻松上手操作系统。

后台界面结合了 MySQL 数据库开发技术,提供可扩展的、可变数据存储使用的数 据库。使用 MySQL 数据库,系统能够存储和使用大量的数据,适应未来的数据增长和 变化。后台界面提供了对数据库的调用功能,使管理员能够方便地进行数据导入、导出 和查询。

无论是前台还是后台,都采用了易于理解和操作的界面设计,让用户能够快速上手 并进行相关操作。即使是不具备基本的电脑使用和维护知识的人,也可以轻松地操作该 系统。这种设计使得系统在易用性和可操作性方面都具备良好的用户体验,为用户提供 了便捷和高效的操作环境。

综上所述,无论从技术可行性还是操作可行性来说开发师生课表自动生成系统都是 可行的。

\chapter{需求分析}

\section{业务流程}

师生课表自动生成系统的组织结构相对单一,主要的角色分别是管理员(教学秘书) 和用户(教师、院系领导、学生等)。教学秘书负责管理端,用户使用普通用户端。

本系统主要业务是课表信息的处理,教学秘书将教务系统中导出的课程表格,添加 到系统中,通过数据导入功能,将表格分解成相应属性存储在后台数据库中。用户根据 需求,使用不同的功能,包括:课表查询、其他查询、统计分析模块,将后台数据通过 不同的约束整理分析,得到相应的内容。在课表查询功能中,用户能够获得教师/教室/ 班级课程表;在其他查询功能中,查询有课相关信息以及无课相关信息;在数据统计分 析功能中,获得表单单项统计数据等以及就原始表单进行的进度、分布、比例等信息分 析。师生课表自动生成系统整体业务流程构造如图 \ref{system_flow} 所示:

 \begin{figure}[htbp]
 \centering
 \includegraphics[width=1\textwidth]{figure/系统业务流程图.png} % 图片文件名
 \caption{系统业务流程图} % 标题自动格式化为"图2-1 示例图片标题"
 \label{system_flow}
 \end{figure}

\subsection{业务需求}

师生课表自动生成系统包含两种角色:教学秘书和用户。教学秘书的主要工作是维 护系统后台数据,用户则是系统的服务主体,能够使用系统课表查询等功能。


\subsection{非功能需求}
\chapter{系统设计}
\section{用例建模}
\subsection{系统功能结构设计}
\subsection{系统用例设计}
\section{静态建模}
\subsection{系统类图的设计}
\subsection{系统数据库表的设计}
\section{动态建模}
\subsection{系统时序图}
\subsection{系统活动图}
\chapter{系统实现}
\section{开发环境与工具}
\subsection{开发语言与重要程序包}
\subsection{开发框架}
\subsection{开发工具}
\subsection{开发硬件环境}
\section{实现困难及修正}
\section{功能模块实现}
\subsection{导入课表的实现}
\subsection{课表查询的实现}
\subsection{查询无课信息的实现}
\subsection{统计分析的实现}
\chapter{系统测试}
\section{测试目的与方法}
\section{测试过程}
\section{测试结论}
\chapter{总\quad 结}





\chapter{表的格式示例(测试用)}

正文小四号宋体,首行缩进2字符,表采用三线表在正文中的常用格式如表2-1所示,请参考使用。

表名设置为宋体五号,居中,位于表上,表内文字设置为宋体五号,垂直居中。

\begin{table}[htbp]
\centering
\vspace{0.5\baselineskip}
\caption{示例表格}\label{tab:example}
\begin{tabular}{cc} % 两列左对齐
\toprule
本质 & 过程 \\
\midrule
途径或方法 & 规划、实施、控制 \\
目标 & 效率、成本效益 \\
活动或作业 & 流动与储存 \\
处理对象 & 原材料、在制品、产成品、相关信息 \\
范围 & 从原点(供应商)到终点(最终顾客) \\
目的或目标 & 适应顾客的需求(产品、功能、数量、质量、时间、价格) \\
\bottomrule
\end{tabular}
\end{table}




 \begin{figure}[t]
 \centering
 \includegraphics[width=0.8\textwidth]{figure/example.png} % 图片文件名
 \caption{示例图片标题} % 标题自动格式化为"图2-1 示例图片标题"
 \end{figure}


测试参考文献使用\cite{WXHK202322015},\cite{CXYY202331031}

\cite{王亚平2006}\cite{Sjostrand:2006za}

这是一个行内公式 $E=mc^2$,位于文字之间。

\[
E=mc^2
\]

% 或使用



% ====================== 分割线=========================
                 % 载入章节内容
\backmatter                             	% 关闭章节序号,对页码没有影响

%% -------------------- 谢辞 -------------------- 	
\chapter{谢\quad  辞}

非常感谢王晗、贾慧敏、李东艳、张雪老师对我的指导和支持。在我进行课程设计 和研究过程中,老师们一直给予了我宝贵的建议和指导,让我能够更好地理解课题背景 和设计要求。老师们在课题开发的整个过程中提供了专业的知识和深入的见解,帮助我 解决了许多难题和困惑。在他们的指导下,我学到了很多关于系统设计和开发的实用技 能和经验,这对我今后的学习和职业发展都具有重要的影响。

在此,我由衷地感谢王晗、贾慧敏、李东艳、张雪四位老师对我的指导和帮助。老 师们的专业知识、耐心指导和关心将成为我学术道路上宝贵的财富。我将倍加珍惜这段 宝贵的学习经历,并努力将所学知识应用到实践中,为实现个人的成长和发展做出更大 的努力。再次感谢王晗、贾慧敏、李东艳、张雪四位老师的支持和鼓励!


感谢自己,我是奶龙,我很棒!

我会继续加油的!

我是奶龙,哈哈哈哈                  % 导入谢辞


%% -------------------- 参考文献 -------------------- 
\renewcommand*{\bibfont}{\zihao{5}\songti}
 \printbibliography[title=参考文献,heading=bibintoc] % 导入参考文献


\end{document}
