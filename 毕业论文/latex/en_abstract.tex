\chapter*{\bfseries ABSTRACT}
\addcontentsline{toc}{chapter}{ABSTRACT}  % 添加到目录
\vspace*{0.5\baselineskip}

The educational administration system has been widely applied in major universities, which has improved the efficiency of course scheduling and provided many high-quality services for teachers and students. However, while the educational administration system brings many conveniences, there are still some problems: the course schedules exported from the school's educational administration system are complex and have poor readability, and generating concise and concise teacher and student (class) schedules, conducting statistical analysis, visualization, and other work consumes a lot of manpower and time.

This paper investigates and studies the relevant literature and practical cases of course schedule generation technology in recent years, analyzes the existing problems from multiple perspectives, and designs the system's business requirements. The system uses the UML modeling method for analysis, and uses use case diagrams, sequence diagrams, class diagrams, etc. to describe the design of various levels of the system, and uses the UML diagrams as an important basis for subsequent development. The system is implemented as a desktop application, and Visual Studio 2019 is chosen as the main development tool, using the WPF+MVVM framework for development. The paper includes a detailed description of the core functional code. During the implementation process, white-box testing is used to ensure the correctness and availability of each logical path. Finally, the overall system is tested, the test environment and system test results are introduced, and the future of the system is described.

The automatic generation system of teacher and student course schedules converts and organizes the original course schedule data to generate concise class, teacher, and classroom schedules, and provides batch query, statistics, analysis, visualization and other functions. It helps teachers and educational administrators to view and process course schedule information more easily, facilitating teachers' teaching work and educational administrators' management work.


\vspace*{1.0\baselineskip}
\noindent
{\fontsize{10.5pt}{12pt}\selectfont \textbf{Key words:}academic affairs system; timetable generation; WPF;UML} % Keywords:} 之后填写英文关键词




\clearpage									% 跳到目录下一页

