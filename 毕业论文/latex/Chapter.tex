\chapter{绪\quad 论}

\section{系统开发背景}

现如今,各式教务系统在各大高校得到了越来越广泛地应用,这些系统极大地提高 了课程排课的准确度和效率,为学校工作人员节省了大量宝贵的时间。通过系统操作, 我们可以轻松地得到合理的排课安排,并且系统能够将大量的课程信息进行有效地整合, 大大减少了课程冲突和效率低下的问题。然而,尽管教务系统带来了诸多便利,但也存 在一些问题。例如,导出的课程表格内容往往十分繁杂,不易于阅读和理解表格中的数 据。此外,将教师课表和学生课表拆分出来的工作量大、效率低,不利于后续根据表格 对教师以及学生信息进行统计分析。为了解决这些问题,需要开发一款能够将教务表格 导入并自动生成简明美观的师生课表的系统。这样的系统将使教务工作更加高效、准确, 同时也为教师和学生提供了更方便的查询和统计分析功能。通过使用简明扼要的课表, 我们可以快速了解每个教师和学生的课程安排情况,及时发现并解决潜在的问题。此外, 此类系统还可以根据用户的个性化需求,定制不同的课表格式和展示方式,以满足用户 的需求。这类系统在应用环境中一般被称为教学辅助系统。

教学辅助软件在现代教育中发挥着越来越重要的作用。尤其是课表转换和分析功能, 为学校、教师和学生提供了许多便利和优势。

课表转换是指将原始的课程安排数据转换为更直观和易于理解的形式。通过教学辅 助软件,将教务系统输出的、繁琐的课程表单转换为直观的、易于阅读的课表格式。这 极大地提高了教师和学生查看和理解课程安排的效率。根据教师、教室和班级等信息自 动化生成课表,从而减少手动制作课表的错误和繁琐过程。

课表分析是指对课程安排数据进行深入分析,获取有关教学活动的信息并进行处理。 校内教务人员可以通过教学辅助软件的课表分析功能,更好地了解到教师的授课情况, 合理安排教师资源,确保教学质量。同时帮助教师明确各学期教学任务、授课时间等, 为教师教学提供依据。

\section{系统开发意义}

推动教务系统的发展,实现课表可视化、便捷化,能帮助教师更好地进行教学管理。 本课题根据大连交通大学教务系统输出的课程表单的实际情况,以及校内教师们的需求, 研究解决繁杂课表分解、师生课表可视化等问题的方法,并提供统计分析数据等数据处 理问题的解决方案。在此基础上,将问题转化为有条理的、符合程序设计流程的用户需 求,并产出清晰的数据流程。从而使教务系统课表更加简明清楚、易于查找,将教务管 理者从复杂的教务管理业务中解脱出来,为智慧校园建设添砖加瓦,也为学校教师提供 更好的服务与管理。

设计一款健壮性高、稳定性高、抗干扰性强、分析能力强的数据处理系统。通过使 用本系统,能够将教务系统中输出的课表进行分解整理,成为教师课表、班级课表和教室课表。通过查询功能,能够指定教师名称、课程名称或班级名称等关键词,查询其对 应的课程安排以及教室分配。通过统计分析功能,了解不同的教师在本学年或学期的教 学时间分配情况,班级分布情况,教学地点分配情况等相关的数据分析结果。也可以进 行关键词筛选,进行多重查询,了解在同一时间段,不同教师的课程安排以及教室情况 等等。为教师提供便利,解决教学生活中遇到的难题。

\section{可行性分析}

目前课表自动生成系统的相关研究较为丰富,研究成果较多,本课题的研究目的主 要是为大连交通大学教务系统产出的学年或学期课程表进行分解,成为个性化、易于查 看、简单美观的教师课表、班级课表和教室课表。并且辅助学校教师教学,增加统计分 析功能,让数据更加直观、可视化。在完成上述功能的同时尽可能保证系统安全稳定。

\subsection{技术可行性}

计划用 Visual Studio 2019 对系统主体进行开发。使用 Visual Studio 2019 具有很多 优势,主要包括以下几点:

(1)统一的集成开发环境:Visual Studio 2019 提供了一个统一的开发环境,能够 在一个地方处理多种编程语言和项目。不需要在不同的工具和编辑器之间切换,从而节 省了时间和精力。

(2)强大的调试功能:Visual Studio 2019 内置了先进的调试功能,快速定位和修 复代码中的错误。在制作系统时能够设置断点、监视变量的值、查看堆栈跟踪等,这些 工具使得调试过程更加高效和准确。

(3)丰富的工具和库:Visual Studio 2019 提供大量的工具和库,用于简化开发过 程、丰富应用程序的功能。运用代码自动完成、重构工具等各种扩展和插件,可以更快 地编写高质量的代码。

(4)强大的图形界面设计器:对于界面开发, Visual Studio 2019 的图形界面设计 器将会是一个强大的工具。它提供了直观的界面编辑器,能够将代码直接转化为图形, 并且可以通过改变图形来填充代码,加快界面开发的速度。

(5)开发者社区和资源库:Visual Studio 2019 拥有一个庞大的开发者社区和丰富 的资源库。在系统开发过程遇到困难时,能够尽快从其他开发人员的经验中获取支持、 经验、问题解决办法。

\subsection{操作可行性}

本系统操作界面设计相对简单,分为前台和后台两个主界面: 

前台界面是使用了 Visual Studio 特有的技术,旨在实现人机的交互。这意味着该系 统需要提供一个用户友好的界面,使用户能够轻松地与系统进行交互。通过使用 Visual Studio 的技术,该系统能够实现快速响应和高效的 PC 应用程序。用户可以通过简单的 页面操作,根据自己的需求选择相应的服务或功能。这种设计使得系统易于使用,即使 是对于没有深入计算机知识的人来说,也能够轻松上手操作系统。

后台界面结合了 MySQL 数据库开发技术,提供可扩展的、可变数据存储使用的数 据库。使用 MySQL 数据库,系统能够存储和使用大量的数据,适应未来的数据增长和 变化。后台界面提供了对数据库的调用功能,使管理员能够方便地进行数据导入、导出 和查询。

无论是前台还是后台,都采用了易于理解和操作的界面设计,让用户能够快速上手 并进行相关操作。即使是不具备基本的电脑使用和维护知识的人,也可以轻松地操作该 系统。这种设计使得系统在易用性和可操作性方面都具备良好的用户体验,为用户提供 了便捷和高效的操作环境。

综上所述,无论从技术可行性还是操作可行性来说开发师生课表自动生成系统都是 可行的。

\chapter{需求分析}

\section{业务流程}

师生课表自动生成系统的组织结构相对单一,主要的角色分别是管理员(教学秘书) 和用户(教师、院系领导、学生等)。教学秘书负责管理端,用户使用普通用户端。

本系统主要业务是课表信息的处理,教学秘书将教务系统中导出的课程表格,添加 到系统中,通过数据导入功能,将表格分解成相应属性存储在后台数据库中。用户根据 需求,使用不同的功能,包括:课表查询、其他查询、统计分析模块,将后台数据通过 不同的约束整理分析,得到相应的内容。在课表查询功能中,用户能够获得教师/教室/ 班级课程表;在其他查询功能中,查询有课相关信息以及无课相关信息;在数据统计分 析功能中,获得表单单项统计数据等以及就原始表单进行的进度、分布、比例等信息分 析。师生课表自动生成系统整体业务流程构造如图 \ref{system_flow} 所示:

 \begin{figure}[htbp]
 \centering
 \includegraphics[width=1\textwidth]{figure/系统业务流程图.png} % 图片文件名
 \caption{系统业务流程图} % 标题自动格式化为"图2-1 示例图片标题"
 \label{system_flow}
 \end{figure}

\subsection{业务需求}

师生课表自动生成系统包含两种角色:教学秘书和用户。教学秘书的主要工作是维 护系统后台数据,用户则是系统的服务主体,能够使用系统课表查询等功能。


\subsection{非功能需求}
\chapter{系统设计}
\section{用例建模}
\subsection{系统功能结构设计}
\subsection{系统用例设计}
\section{静态建模}
\subsection{系统类图的设计}
\subsection{系统数据库表的设计}
\section{动态建模}
\subsection{系统时序图}
\subsection{系统活动图}
\chapter{系统实现}
\section{开发环境与工具}
\subsection{开发语言与重要程序包}
\subsection{开发框架}
\subsection{开发工具}
\subsection{开发硬件环境}
\section{实现困难及修正}
\section{功能模块实现}
\subsection{导入课表的实现}
\subsection{课表查询的实现}
\subsection{查询无课信息的实现}
\subsection{统计分析的实现}
\chapter{系统测试}
\section{测试目的与方法}
\section{测试过程}
\section{测试结论}
\chapter{总\quad 结}





\chapter{表的格式示例(测试用)}

正文小四号宋体,首行缩进2字符,表采用三线表在正文中的常用格式如表2-1所示,请参考使用。

表名设置为宋体五号,居中,位于表上,表内文字设置为宋体五号,垂直居中。

\begin{table}[htbp]
\centering
\vspace{0.5\baselineskip}
\caption{示例表格}\label{tab:example}
\begin{tabular}{cc} % 两列左对齐
\toprule
本质 & 过程 \\
\midrule
途径或方法 & 规划、实施、控制 \\
目标 & 效率、成本效益 \\
活动或作业 & 流动与储存 \\
处理对象 & 原材料、在制品、产成品、相关信息 \\
范围 & 从原点(供应商)到终点(最终顾客) \\
目的或目标 & 适应顾客的需求(产品、功能、数量、质量、时间、价格) \\
\bottomrule
\end{tabular}
\end{table}




 \begin{figure}[t]
 \centering
 \includegraphics[width=0.8\textwidth]{figure/example.png} % 图片文件名
 \caption{示例图片标题} % 标题自动格式化为"图2-1 示例图片标题"
 \end{figure}


测试参考文献使用\cite{WXHK202322015},\cite{CXYY202331031}

\cite{王亚平2006}\cite{Sjostrand:2006za}

这是一个行内公式 $E=mc^2$,位于文字之间。

\[
E=mc^2
\]

% 或使用



% ====================== 分割线=========================
