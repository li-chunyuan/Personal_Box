%%%%%%%%%%%%%%%%%%%%%%%%%%%%%%%%%%%%%%%%%%%%%%%%%%%%%%%%%%%%%%%%%%
%   使用说明:使用XeLatex(2019+)进行编译,建议关闭拼写检查
%	模版内容结构
%	- 格式相关:main.tex(学生无需处理操作)
%	- 论文内容:/latex目录下,包含封面(cover.tex)、中文摘要(cn_abstract.tex)、英文摘要(en_abstract)、正文(Chapter.tex)、谢辞(Thanks.tex)
%%%%%%%%%%%%%%%%%%%%%%%%%%%%%%%%%%%%%%%%%%%%%%%%%%%%%%%%%%%%%%%%%%

%=============================文档类型==============================
%	twoside命令:设置为双面排版,左右页边距会根据奇偶页自动调整
%	12pt:正文字体大小
%	openany:新的一章可以出现在左右两侧,方式ctexbook类型使得空白页出现
%	AutoFakeBold:实现粗体形式宋体
%
%=============================宏包调用==============================
\usepackage{wallpaper}	% 封面背景包
\usepackage{amsmath,mathtools,amsthm,amsfonts,amssymb,bm}	% AMS包
\usepackage{color}  % 字体背景颜色包
\usepackage{xeCJK}  % 中文字体配置包
\usepackage{ulem} % 下划线增强
\usepackage{setspace} % 行距控制
\usepackage{calc} % 计算长度
\usepackage{ulem}
\usepackage{setspace}
\usepackage{varwidth}  % 新增可变宽度盒子支持
\usepackage{xparse}    % 新增高级命令定义
\usepackage{soul}  % 更强大的下划线支持
\usepackage{tabularx}
\usepackage{setspace}
\usepackage{listings}
\usepackage{multicol} % 多列布局包

%==========================页面边距设置=============================
\usepackage[a4paper,left=25mm,right=25mm,top=25mm,bottom=25mm,headheight=15mm,headsep=4mm ,footskip=17.5mm]{geometry}	

%	页边距:上2.5cm,下2.5cm,左2.5cm,右2.5cm。
%	页眉顶端距离:1.5cm
%	页脚地段距离:1.75cm

%==========================页眉页脚设置=============================
\usepackage{fancyhdr}	% 页眉页脚包
\fancypagestyle{djtu}{
	\fancyhf{}		% 清空所有定义
	\fancyfoot[CE,CO]{\zihao{-5} \songti \thepage}      % 设置页脚为当前页码,居中,小五,数字1,2,3等
	\fancyhead[CE]{\zihao{-5} \heiti 大连交通大学 2026 届本科生综合毕业设计(论文)}  % 偶数页眉,居中,小五,黑体 
	\fancyhead[CO]{\zihao{-5} \heiti 大连交通大学 2024 届本科生综合毕业设计(论文)}  % 奇数页眉,居中,小五,黑体
	}
 \fancypagestyle{plain}{
 	\fancyhf{}		% 清空所有定义
 	\fancyfoot[CE,CO]{\zihao{-5} \songti \thepage}      % 设置页脚为当前页码,居中,小五,数字1,2,3等
 	\fancyhead[CE]{\zihao{-5} \heiti 大连交通大学 2026 届本科生综合毕业设计(论文)}  % 偶数页眉,居中,小五,黑体 
 	\fancyhead[CO]{\zihao{-5} \heiti 大连交通大学 2026 届本科生综合毕业设计(论文)}  % 奇数页眉,居中,小五,黑体
     }
% 定义摘要部分的页面样式:只显示页码,无页眉
\fancypagestyle{abstract}{
	\fancyhf{}		% 清空所有定义
	\fancyfoot[CE,CO]{\zihao{-5} \thepage}      % 设置页脚为当前页码,居中,小五
	}

%==========================字体配置设置=============================
\usepackage{fontspec}	% 字体配置宏包
\setmainfont{Times New Roman} % 设置英文字体为新罗马字体

%==========================段落相关设置=============================
\usepackage{setspace} 	% 段落行距包
\usepackage{indentfirst} %段落首行缩进命令包
\usepackage{lipsum}
\setlength{\baselineskip}{20pt}



% 行距的设定为20磅,需在\begin{document}中补充设置
\raggedbottom			% 防止latex排版导致段落间距变大
\setlength{\parindent}{2em}		% 首航缩进距离设置为2em,表示两个字符

%==========================章节格式设置=============================
\usepackage{ctex}
\ctexset{ % 用来定制正文里的标题格式
	chapter = {%
		name = {第,章},
		number = \chinese{chapter},              % 用中文数字显示章节号
		format = {\heiti \zihao{3} \centering},       % 设置章节标题为3号黑体且居中
		beforeskip = 0pt,			        % 段前行距0行	
		afterskip = 20pt,			        % 段后行距0行
		fixskip = true				        % 设置固定间距为true,抑制标题前后的多余间距
	},
	section = {%
		number = \normalfont\arabic{chapter}.\normalfont\arabic{section} \hspace{-1em}, % 编号格式为1.1,编号与题目名之间的空格
		format = {\heiti\zihao{4}\raggedright},   % section格式添加一条:左对齐
        beforeskip = 0pt,	% 段前行距0行
        afterskip = 0pt,	% 段后行距0行
	},
	subsection = {%
		number = \normalfont\arabic{chapter}.\normalfont\arabic{section}.\normalfont\arabic{subsection} \hspace{-1em},% 编号格式为1.1.1,编号与题目名之间的空格
		format = {\heiti\zihao{4}\raggedright},    % subsection格式添加一条:左对齐
        beforeskip = 0pt,	% 段前行距0行
        afterskip = 0pt		% 段后行距0行
	},
}

%==========================目录格式设置=============================
\usepackage{titletoc}		                % 目录定制包
\renewcommand{\contentsname}{目\quad 录}		% “目录”两字之间空X格

\titlecontents{chapter}
[0em] % 缩进量
{\bfseries\songti\zihao{-4}} % 控制字体字号
{\thecontentslabel\hspace{0em}} % 控制章节编号和名称的距离 
{} 
{\titlerule*[3pt]{.}\hspace{-1em}\contentspage} % 控制引导点密度,这里引入\hspace{-1em}的功能是防止页码和引导点之间有空格

\titlecontents{section}
[1em] % 缩进量
{\songti\zihao{-4}} % 控制字体字号
{\thecontentslabel\hspace{1em}} % 控制小节编号和名称的距离 
{} 
{\titlerule*[3pt]{.}\hspace{-1em}\contentspage} % 控制引导点密度,这里引入\hspace{-1em}的功能是防止页码和引导点之间有空格


\titlecontents{subsection}
[2em] % 缩进量
{\songti\zihao{-4}} % 控制字体字号
{\thecontentslabel\hspace{1em}} % 控制子节编号和名称的距离
{} 
{\titlerule*[3pt]{.}\hspace{-1em}\contentspage}% 控制引导点密度,这里引入\hspace{-1em}的功能是防止页码和引导点之间有空格

%==========================参考文献设置=============================
\usepackage[%
	backend=biber,				% 设置使用biber进行编译,也可以使用bibtex,但是功能更少
	style=gb7714-2015,			% 设置风格样式为国家标准gb7714-2015
	sorting=none					% 设置排序按照年份,名字,标题进行排序,若想按照引用顺序排序,将其设置为none即可
	]{biblatex}       			% 参考文献包
\addbibresource{bib/ref.bib}    % 加载参考文献的文件
\usepackage[hidelinks]{hyperref}  % 加载hyperref,实现参考文献引文跳转


%==========================图表环境设置=============================
\usepackage[inline]{enumitem}					% 列表工具包
\usepackage{graphicx,ragged2e}					% 插图工具包
\usepackage{subcaption}							% 子图标题包
\usepackage{bicaption}							% 图片标题包
\setlist{%	设置列表样式
	topsep=0.3em, 			% 列表顶端的垂直空白
	partopsep=0pt, 			% 列表环境前面紧接着一个空白行时其顶端的额外垂直空白
	itemsep=0ex plus 0.1ex, % 列表项之间的额外垂直空白
	parsep=0pt, 			% 列表项内的段落之间的垂直空白
	leftmargin=1.5em, 		% 环境的左边界和列表之间的水平距离
	rightmargin=0em, 		% 环境的右边界和列表之间的水平距离
	labelsep=0.5em, 		% 包含标签的盒子与列表项的第一行文本之间的间隔
	labelwidth=2em 			% 包含标签的盒子的正常宽度;若实际宽度更宽,则使用实际宽度。
}
\graphicspath{figure/}		% 设置图片存放目录
\usepackage{longtable,booktabs} % 跨页长表格
\usepackage{caption}
\captionsetup[table]{skip=0pt} % 设置表格标题与内容之间无间距
\captionsetup[figure]{skip=10pt} % 设置图片标题与内容之间无间距
\renewcommand{\arraystretch}{1.5} % 扩大表格每行文字距离
\DeclareCaptionLabelSeparator{quadspace}{\quad}  % 定义分隔符为\quad
\captionsetup[table]{
    labelsep=quadspace,                   % 标签与标题之间的分隔符
    justification=centering,              % 标题居中
    font=small,                           % 字体大小
}
\captionsetup[figure]{
    labelsep=quadspace,                   % 标签与标题之间的分隔符
    justification=centering,              % 标题居中
    font=small,                           % 字体大小
}
% 修改表格编号格式为 "章号-表格序号"
\renewcommand{\thetable}{\arabic{chapter}-\arabic{table}}
% 修改图片编号格式为 "章号-图片序号"
\renewcommand{\thefigure}{\arabic{chapter}-\arabic{figure}}

\usepackage{enumitem} % 必需宏包(用于(1)、(2)...)
\usepackage{pifont}   % 提供带圈数字符号

\newlist{customenum}{enumerate}{1}
\setlist[customenum]{
    label=(\arabic*),   % 标签格式
    leftmargin=*,       % 左对齐
    labelwidth=1em,     % 标签宽度
    align=left          % 对齐方式
}

\newlist{customsubenum}{enumerate}{1}
\setlist[customsubenum]{
    label=\ding{172},   % 使用CJK带圈数字符号
    leftmargin=2em,     % 调整缩进量
    labelwidth=1.5em,
    align=left
}

% 禁用mainmatter的分页,删除目录后一页产生的空白页
\makeatletter
\renewcommand{\mainmatter}{%
  \clearpage
  \@mainmattertrue
  \pagenumbering{arabic}
}
\makeatother



