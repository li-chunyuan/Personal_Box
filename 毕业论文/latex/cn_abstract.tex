\chapter*{摘\quad  要}
\addcontentsline{toc}{chapter}{摘\quad  要}  % 添加到目录
\newcommand{\keywords}[1]{\text{\heiti #1}}
\vspace*{0.5\baselineskip}

现如今,教务系统在各大高校得到广泛地应用,这些系统提高了课程排课效率,也 为师生提供了许多优质的服务。但教务系统带来诸多便利的同时仍然存在着一些问题: 目前从学校教务管理系统中导出的师生课表,内容繁杂可读性较差,为生成简明扼要的 教师、学生(班级)课表,开展统计、分析、可视化等工作消耗大量的人力和时间。

本文通过调查研究近年来课表生成等技术的相关文献及实际案例,对现有问题进行 多方面的分析进而设计出系统的业务需求。系统采用UML建模方法进行分析,使用用例 图、时序图、类图等描述系统各个层面的设计,并以UML图作为后续开发的重要依据。 系统以桌 面 应用程 序 实现, 选 择Visual Studio 2019作 为主 要 开 发工 具, 采用 了 WPF+MVVM框架进行开发。 论文中包含了核心功能代码的详细描述。在实现过程中通 过白盒测试确保每一个逻辑通路的正确可用性。最后对系统整体进行测试,介绍了测试 环境以及系统测试结果。并对系统未来进行展望。

师生课表自动生成系统通过转换和整理原始课表数据,生成简明扼要的班级、教 师、教室课表,并提供批量查询、统计、分析、可视化等功能。帮助教师和教务人员更 轻松地查看和处理课表信息,方便教师开展教学工作、教务人员进行管理工作。


\vspace*{1.0\baselineskip}
\noindent
{\fontsize{10.5pt}{12pt}\selectfont \keywords{关键词:} 教务系统;课表生成;WPF;UML}% 关键词:} 之后填写中文关键词

\clearpage						% 跳到目录下一页

