
\documentclass[12pt]{ctexart}

\usepackage{geometry}
\geometry{a4paper,margin=2.2cm}

\usepackage{tikz}
\usetikzlibrary{arrows.meta,positioning}

\usepackage{enumitem}
\setlist[itemize]{leftmargin=2em}
\setlist[enumerate]{leftmargin=2em}

\usepackage{hyperref}
\hypersetup{
    colorlinks=true,
    linkcolor=blue,
    urlcolor=blue
}

\title{嵌入式人生路线图(Gap Year 版本)}
\author{Grant Lee}
\date{\today}

\begin{document}
\maketitle

\section*{总体时间轴(概览)}

\begin{center}
\begin{tikzpicture}[
    >=Stealth,
    timeline/.style={thick},
    phase/.style={rectangle,rounded corners,draw,align=center,inner sep=4pt,font=\small},
    >=Stealth
]
    % 横轴
    \draw[timeline] (0,0) -- (15,0);

    % 刻度与标签
    \foreach \x/\lab in {
        0/Gap Year 起点,
        2/0--3 月,
        6/3--9 月,
        9/9--12 月,
        11/毕业后第 1 年,
        13/毕业后第 2 年,
        15/研一--研三
    }{
        \draw (\x,0.1) -- (\x,-0.1);
        \node[below=3pt,font=\scriptsize,align=center] at (\x,-0.1) {\lab};
    }

    % 各阶段块
    % 0-3 月:C 语言 + 写书
    \node[phase,above=6mm of {(1,0)}] (p1) {0--3 月\\C 语言教材\\(写完 Part1--Part4)};
    \draw[->] (1,0) -- (p1.south);

    % 3-9 月:嵌入式实习
    \node[phase,above=12mm of {(4.5,0)}] (p2) {3--9 月\\嵌入式实习 6 个月\\工具链 \& 工程经验};
    \draw[->] (4.5,0) -- (p2.south);

    % 9-12 月:回校读大四
    \node[phase,above=6mm of {(8,0)}] (p3) {9--12 月\\回校读大四\\强化模电 / 数电};
    \draw[->] (8,0) -- (p3.south);

    % 毕业后第 1 年:考研 + 实习
    \node[phase,above=12mm of {(11,0)}] (p4) {毕业后第 1 年\\考研\#1 + 嵌入式实习\\打磨基础与项目};
    \draw[->] (11,0) -- (p4.south);

    % 毕业后第 2 年:考研冲刺
    \node[phase,above=6mm of {(13,0)}] (p5) {毕业后第 2 年\\考研\#2 备战\\保持工程手感};
    \draw[->] (13,0) -- (p5.south);

    % 研一--研三:深入方向
    \node[phase,above=12mm of {(15,0)}] (p6) {研一--研三\\深入嵌入式/系统方向\\目标月薪 30k--40k};
    \draw[->] (15,0) -- (p6.south);

\end{tikzpicture}
\end{center}

\bigskip

\section*{阶段拆解规划}

\subsection*{阶段 1(0--3 月):写完 C 语言教材 = 打完底层内功}

\paragraph{目标:}
\begin{itemize}
    \item 以「写书」的形式系统学完 C 语言;
    \item 写出一份可反复复用的个人 C 语言教材,用作之后所有项目与工作的参考手册。
\end{itemize}

\paragraph{建议结构(可按需调整):}
\begin{enumerate}[label=\textbf{Part \arabic*:},wide=0pt]
    \item \textbf{基础部分(你正在写,计划本周完成)}
    \begin{itemize}
        \item C 程序基本结构、编译流程;
        \item 变量与常量、字面量;
        \item 各类数据类型(整数、浮点、字符、枚举、指针类型);
        \item 比特、字节、补码、IEEE~754、类型的本质;
        \item 小结:\emph{类型 = 对比特模式的解释方式}。
    \end{itemize}

    \item \textbf{核心语法与指针体系}
    \begin{itemize}
        \item 表达式与运算符、控制语句(if/for/while/switch);
        \item 数组与字符串的内存布局;
        \item 指针基础:地址、解引用、一维数组 \& 指针;
        \item 指针与数组的等价形式:\verb|a[i] <=> *(a + i)|;
        \item 多级指针、函数指针(点到为止即可)。
    \end{itemize}

    \item \textbf{组合与抽象能力}
    \begin{itemize}
        \item 函数与参数传递(值传递 vs. 地址传递);
        \item \texttt{struct}/\texttt{union}、位段、结构体内存对齐;
        \item 使用结构体实现简单数据模型(如学生信息、任务描述等)。
    \end{itemize}

    \item \textbf{工程向部分(实习前打底)}
    \begin{itemize}
        \item 动态内存管理:\texttt{malloc/free/realloc},常见内存错误;
        \item 文件~IO:\texttt{fopen/fread/fwrite/fprintf};
        \item 多文件与头文件设计:\texttt{.h + .c} 的组织方式;
        \item 简单构建流程:\texttt{gcc}、\texttt{Makefile}(入门级即可)。
    \end{itemize}
\end{enumerate}

\paragraph{刻意跳过(暂不写 / 只点到为止):}
\begin{itemize}
    \item C 标准的细枝末节(完整的 UB 列表、表达式求值顺序的极端细节);
    \item C11/C17 的多线程与原子操作;
    \item 编译器内部实现(如 SSA、寄存器分配等)。
\end{itemize}

\subsection*{阶段 2(3--9 月):嵌入式实习 6 个月 = 打开工具链与工程能力}

\paragraph{目标:}
\begin{itemize}
    \item 用 6 个月换取真正的工程经验,而不是工资;
    \item 熟悉主流 MCU(如 STM32)的完整开发流程;
    \item 在真实项目中巩固 C 语言和 Part~1--4 内容。
\end{itemize}

\paragraph{重点收获(写在计划里,方便回顾):}
\begin{itemize}
    \item 编译工具链:Keil / MDK / GCC / CMake / Make 等;
    \item 硬件相关:GPIO、UART、SPI、I\textsuperscript{2}C、ADC、定时器、NVIC;
    \item 调试技能:单步、断点、观察寄存器和内存;
    \item 代码管理:Git 工作流、代码规范、Code Review;
    \item 初步接触 RTOS(如 FreeRTOS)、简单任务调度逻辑。
\end{itemize}

\subsection*{阶段 3(9--12 月):回校读大四 + 强化模电/数电}

\paragraph{目标:}
\begin{itemize}
    \item 补强「模拟电路」「数字电路」基础,使自己具备\emph{软硬皆通}的嵌入式素质;
    \item 在课程项目中有意识地选择与嵌入式 / 硬件相关的题目;
    \item 整理本科学习阶段的知识,开始思考考研方向。
\end{itemize}

\subsection*{阶段 4(毕业后第 1--2 年):考研窗口 + 实习穿插}

\paragraph{第 1 年:}
\begin{itemize}
    \item 以考研为主线,构建系统的数学与专业基础复习计划;
    \item 争取找到一份与嵌入式 / C 语言相关的兼职或短期实习,保持工程手感;
    \item 根据初次考试结果,调整学校和方向定位。
\end{itemize}

\paragraph{第 2 年:}
\begin{itemize}
    \item 若第 1 年未成功,则第 2 年作为冲刺年,专注弥补短板;
    \item 同样保持少量工程实践(避免毕业后「只有理论,没有手感」)。
\end{itemize}

\subsection*{阶段 5(研一--研三):方向深入 + 项目沉淀}

\paragraph{目标:}
\begin{itemize}
    \item 选定「嵌入式系统 / 操作系统 / 驱动 / 工业控制」等相关方向深入;
    \item 至少完成 2--3 个具有代表性的工程项目(可在 GitHub 上公开展示);
    \item 毕业时具备月薪 30k--40k 的综合实力(底层扎实 + 工程经验 + 项目说服力)。
\end{itemize}

\bigskip
\noindent\textbf{备注:}本计划是一个「可迭代」的版本,实际执行中可根据自身状态、机会和外界环境进行微调。核心思想是:
\begin{center}
    \emph{先打基础(C \& 电路)→ 再上工程战场(实习)→ 再系统提升(考研 \& 研究生)。}
\end{center}

\end{document}
