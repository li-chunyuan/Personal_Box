% =============================================================
% ②  版面与图文排版类(建议尽量靠前加载,确保后续宏包获得正确尺寸)
% =============================================================
\usepackage[top=25mm,bottom=25mm,left=25mm,right=25mm]{geometry}
% geometry —— 页面边距设置      作用:统一设置页边距,常用于论文/报告规范化排版。    备注:最好最先加载,避免与后续版面相关宏包产生尺寸冲突

\usepackage{graphicx}
% graphicx —— 插图支持      插入图片;原生支持 PNG/JPG/PDF(不支持 WebP)      用法:\includegraphics[width=\linewidth]{fig.pdf}

\usepackage{float}   % float —— 提供 [H] 定位方式,使 table/figure 绝对不浮动
% =============================================================
% ③  关键配件支持类宏包(颜色 / 符号 / 微排版等)
% =============================================================
\usepackage{xcolor}
% 颜色支持——定义/使用颜色;为 listings、表格等提供配色基础      说明:支持预定义颜色名、RGB/CMYK、混色(如 gray!70!black)

\usepackage{pifont}
% pifont——符号字库,提供特殊符号,如对号、叉号、圆圈数字等。     示例:\ding{172} 输出“①”;\ding{51} 是“对号”

\usepackage{microtype}
% microtype——微排版           注意:不同引擎与语言环境支持略有差异,但通常“开箱即用”
% 作用:改进字偶距/字距/断行,整体视觉更均衡

% =============================================================
% ④ 题注格式与列表环境
% =============================================================
\usepackage{caption}
% 题注格式美化——统一图片/表格/代码等题注的字号、加粗、间距等      说明:与 listings 结合可获得一致的“代码 x:标题”样式
\captionsetup{font=small,labelfont=bf} % 小号题注 + 加粗“标签名”

\usepackage{subcaption}%子图支持
% 子图支持——在一个 figure 环境内插入多个子图,并为每个子图添加独立题注

\usepackage{enumitem}   % 自定义列表样式
%\begin{enumerate}[label=\ding{\numexpr171+\arabic*}]
%  \item 第一项
%  \item 第二项     ①②③④⑤…
%  \item 第三项
%\end{enumerate}
%
% \begin{enumerate}[label={(\arabic*)}]
%   \item 第一项
%   \item 第二项    (1)(2)(3)
%   \item 第三项
% \end{enumerate}

% =============================================================
% ⑤ 表格增强类
% =============================================================
\usepackage{array}%表格列格式增强宏包      % 改进 LaTeX 原生的 tabular 环境,可以定义自定义列类型      p{3cm} 表示固定宽度 3cm 的文本列(支持自动换行)

\usepackage{tabularx}%占满整行的表格           
% 使用 X 列型让表格自动填满页面宽度。
\newcolumntype{C}{>{\centering\arraybackslash}X}  
% 定义新列类型 C:在 tabularx 中表示“自动伸缩且内容居中”的列

\usepackage{booktabs}%表格横线      
% \toprule  表示表格最上面的粗线
% \midrule  表示表头和内容之间的中线
% \bottomrule 表格底部的粗线
% 专业表格横线     提供三个表格命令:\toprule、\midrule、\bottomrule。

% =============================================================
% ⑥ 代码类
% =============================================================
%  - listings:用于插入和高亮显示程序代码
\usepackage{listings}
\renewcommand{\lstlistingname}{代码}% 标题改为"代码"
\lstdefinestyle{cstyle}{
  language=C,
  basicstyle=\ttfamily\small,% 代码:等宽字体、小号字体
  numbers=left,% 左侧显示行号
  numberstyle=\scriptsize\color{gray},%行号样式
  numbersep=8pt,%行号与代码之间的间距
  commentstyle=\color{gray!70!black},% 注释样式:灰色
  showstringspaces=false,% 不显示字符串中的空格(默认会用下划线表示)
  tabsize=4,% 设置 Tab 宽度为 4 个空格
  frame=single, % 给代码添加单线边框
  rulecolor=\color{teal!70!blue!30}, % 清新蓝绿色边框
  frameround=tttt,% 四个角都采用圆角(t=top,b=bottom)
  columns=fullflexible,  % 列宽自动调整,使字符不被压缩
  keepspaces=true % 保持代码中的空格格式
}
\lstset{style=cstyle}% 将上面定义的 cstyle 样式设为默认

% =============================================================
% ⑦ 数学公式与数学环境支持类
% =============================================================
\usepackage{amsmath}
% amsmath —— 提供 aligned、gather、split、cases 等高级数学排版环境
% 作用:扩展 LaTeX 数学模式功能,是几乎所有论文必备宏包
% 提示:尽量在使用数学环境前加载;避免与某些旧数学宏包混用

\usepackage{amssymb}
% amssymb —— American Mathematical Society 提供的数学符号扩展
% 作用:提供 \mathbb、\mathfrak、更多数学符号(如 \leqslant)
% 说明:不与 amsmath 冲突,通常一起使用

\usepackage{mathtools}
% mathtools —— 对 amsmath 的增强补丁
% 作用:提供更强大的对齐、间距控制;补充 amsmath 不足;
% 常用示例:\coloneqq、\mathclap、\adjustlimits 等
% 说明:必须在 amsmath 之后加载

% 数学字体与特殊样式(可选)
%\usepackage{bm}
% bm —— 粗体数学符号(加粗向量/矩阵),用法 \bm{x}
%
%\usepackage{mathrsfs}
% mathrsfs —— 花体(script)数学字母:\mathscr{F}
% 注意:和 \mathcal 不同,呈现更柔和的“花体风格”
%
% 数学环境小规则(可选建议)
% \numberwithin{equation}{section}
% 让公式编号随 section 编号,例如 2.1、2.2 …

% =============================================================
% ⑧ 超链接与智能引用类(加载顺序:hyperref 在前,cleveref 在后)
% =============================================================
\usepackage{hyperref}%超链接/目录可点击
\hypersetup{
  colorlinks=false,   % 不用彩色文字
  pdfborder={0 0 0}   % 去掉红框
}

\usepackage{cleveref}%智能引用
%----------------------代码-------------------------------------
\crefname{lstlisting}{代码}{代码}% 标题改为"代码"
\providecommand{\lstlistingautorefname}{代码}% 自动引用时标题改为"代码"

%----------------------图片-------------------------------------
\renewcommand{\figurename}{图}% 改 caption 名称为“图”
\renewcommand{\figureautorefname}{图}% 自动引用时标题改为“图”

%----------------------表格-------------------------------------
\renewcommand{\tablename}{表}% 改 caption 名称为“表”
\renewcommand{\tableautorefname}{表}% 自动引用时标题改为“表”

%----------------------章节-------------------------------------
\renewcommand{\sectionautorefname}{章}
\renewcommand{\subsectionautorefname}{节}
\renewcommand{\subsubsectionautorefname}{小节}


% =============================================================
% ⑨ 章节编号与目录深度
% =============================================================
% ③ 三级标题(section/subsection/subsubsection)
\setcounter{tocdepth}{2}    % 目录显示到二级标题
\setcounter{secnumdepth}{3} % 标题编号到三级标题


% \vspace{1em}
%\noindent\textbf{(1)~标题(Well-defined)}\par\vspace{0.5em}