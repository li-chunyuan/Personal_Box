\section{C语言程序的基本结构}
\subsection{最小示例 \texttt{HelloWorld}}%\texttt:等宽字体显示
\begin{lstlisting}[caption={最小 C 程序示例},label={lst:hello}]

  #include <stdio.h>          // 1) 预处理指令:包含头文件

int main(void) {            // 2) 主函数:程序从这里开始执行
    printf("Hello, World!\n"); // 3) 调用库函数进行输出
    return 0;               // 4) 返回状态码 0:正常结束
}
\end{lstlisting}

\subsection{\#include 与头文件}
\begin{itemize}%用于创建无序列表(项目符号列表)的环境
  \item \texttt{\#include }:\#include是预处理指令,在编译开始
  前,编译器会把被包含的文件内容“复制粘贴”进来。
  \item \texttt{stdio.h} :  \texttt{stdio.h} 中定义了常用的输入输出函数
  (如 printf, scanf)。如果事先不在预处理阶段声明定义,后续却在代码中使用了
  stdio.h 中的函数,那么编译器就会报错。
  \item 尖括号 \verb|<...>| 在系统路径查找;引号 \verb|"my.h"| 先在当前目录再到系统路径查找。
\end{itemize}

\subsection{\texttt{main} 函数(程序入口)}
\begin{lstlisting}[caption={常见main函数入口示例},label={lst:simple_main}]
int main(void)
// 或
int main(int argc, char *argv[])
\end{lstlisting}

\vspace{1em}%空出一行距离(当前字体高度)

在 C 语言中,main() 是程序的唯一入口函数,程序的执行从这里开始。

\vspace{1em}%空出一行距离(当前字体高度)


\begin{itemize}
  \item 接受:main函数的参数只能是 void 或命令行参数。究其原因是
  因为main函数是由系统启动代码调用的,系统启动时,系统只知道程序名、命令行参数和环境变量。
  
  \item 返回值:在 C 语言标准(C99、C11、C17、C23)中明确规定,main 函数必须返回一个 int 类型
的值,不能是 char、float、double、void 等其他类型,用来表示程序的执行状态。
\begin{enumerate}[label=\ding{\numexpr171+\arabic*}]% 圆圈编号,自动编号为①②③…
  \item \texttt{return 0}: 表示正常结束。
  \item \texttt{return 非 0}: 表示异常结束,具体值可自定义以表示不同错误类型。
\end{enumerate}

\end{itemize}





\subsection{语句块与分号}
\begin{itemize}
  \item 花括号 \verb|{ ... }| 构成语句块(复合语句),可包含声明与可执行语句。
  \item 大多数 C 语句以分号 \texttt{;} 结尾:表达式语句、声明(含初始化)、\texttt{return} 等。
\end{itemize}

\subsection{\texttt{printf} 与常用占位符}
\begin{lstlisting}[caption={printf 与占位符示例}]
#include <stdio.h>
int main(void) {
    int a = 42; double x = 3.14159; char c = 'A';
    printf("a=%d, x=%.2f, c=%c\n", a, x, c); // 输出 a=42, x=3.14, c=A
    return 0;
}
\end{lstlisting}




\subsection{从源代码到可执行文件:将自然语言翻译为机器语言}

C 程序从源代码到可执行文件一般经历四个主要阶段:\textbf{预处理}、\textbf{编译}、\textbf{汇编} 和 \textbf{链接}。下面以 \texttt{gcc} 为例分别说明,并给出对应命令。为便于工程化,还补充多文件与库的常见用法及常见问题。

\paragraph{(1)预处理(Preprocessing)}
主要工作:展开 \#include、宏替换(\#define)、条件编译(\#if/\#ifdef 等),并去除注释,得到纯 C 源。
\begin{verbatim}
gcc -E hello.c -o hello.i
\end{verbatim}
产物:\texttt{.i} 文件(预处理后的 C 代码)。

\paragraph{(2)编译(Compilation)}
对 \texttt{.i} 做词法/语法/语义分析与优化,生成目标架构的汇编代码。
\begin{verbatim}
gcc -S hello.i -o hello.s      % 也可直接对 .c 使用 -S
\end{verbatim}
产物:\texttt{.s} 文件(汇编源码)。

\paragraph{(3)汇编(Assembly)}
把汇编转为可重定位的目标文件,包含代码段、数据段与符号表。
\begin{verbatim}
gcc -c hello.s -o hello.o      % 也可 gcc -c hello.c -o hello.o
\end{verbatim}
产物:\texttt{.o} 文件(目标文件)。

\paragraph{(4)链接(Linking)}
将若干 \texttt{.o} 与所需库合并,进行符号解析与重定位,加入启动代码,生成可执行文件(Linux 为 ELF,Windows 为 PE)。
\begin{verbatim}
gcc hello.o -o hello
\end{verbatim}

\paragraph{(5)装载与运行(Loading \& Running)}
操作系统加载可执行文件到内存,先进入运行时入口 \texttt{\_start},再调用用户的 \texttt{main()};\texttt{main} 返回值作为进程退出码。

\paragraph{(6)多文件与库的基本用法}
\begin{verbatim}
gcc -c main.c -o main.o
gcc -c util.c -o util.o
gcc main.o util.o -o app              % 链接生成可执行文件

ar rcs libmylib.a foo.o bar.o         % 生成静态库
gcc main.o -L. -lmylib -o app2        % 使用静态库(库放在对象文件之后)

gcc -fPIC -c foo.c -o foo.o
gcc -shared -o libmylib.so foo.o      % 生成共享库
gcc main.o -L. -lmylib -o app3        % 运行时需能找到共享库
\end{verbatim}

\paragraph{(7)常见问题速查}
\begin{itemize}
  \item 头文件找不到:添加包含路径 \texttt{-Ipath}。
  \item 未定义引用(\texttt{undefined reference}):缺少对应对象文件或库,或库链接顺序在前;应将 \texttt{-lxxx} 放在使用到该库的对象文件之后。
  \item 库找不到:添加库路径 \texttt{-Lpath},并确认库名(\texttt{libxxx.a}/\texttt{libxxx.so} 对应 \texttt{-lxxx})。
  \item 位宽/架构不匹配:确认 \texttt{-m32}/\texttt{-m64} 或交叉编译器是否正确。
\end{itemize}

\paragraph{(8)嵌入式特例}
在无操作系统的嵌入式环境中,仍有编译与链接步骤,但会使用启动文件与链接脚本生成固件映像(如 \texttt{.hex}/\texttt{.bin}),\texttt{main()} 通常为永不返回的主循环。

构建流程:
\begin{center}
\texttt{.c 源文件}
$\Rightarrow$ 预处理(展开 \#include/\#define)
$\Rightarrow$ 编译(生成 \texttt{.o/.obj})
$\Rightarrow$ 链接(合并库与目标文件)
$\Rightarrow$ 可执行文件
\end{center}

命令行示例(以 \texttt{gcc} 为例):
\begin{verbatim}
gcc hello.c -o hello   # 生成可执行文件
./hello                # 运行(Windows: hello.exe)
\end{verbatim}
