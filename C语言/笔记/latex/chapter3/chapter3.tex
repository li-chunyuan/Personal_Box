
\newpage

\section{数据类型}


为什么在计算机运算时要指定数据的类型呢?
在数学中,数值是不分类型的,数值的
运算是绝对准确的,例如:78与97之和为175,
1/3的值是0.33333333……(循环小数)。数学是
一门研究抽象问题的学科,数和数的运算都是抽象
的。

而在计算机中,数据是存放在存储单元中的,
它是具体存在的。而且,存储单元是由有限的字节
构成的,每一个存储单元中
存放数据的范围是有限的,不可能存放“无穷大”的数,
也不能存放循环小数。

例如用C程序计算和输出1/3:
程序printf("\%f" , 1.0/3.0);
得到的结果是0.333333, 只能得到6位小数,而不是
无穷位的小数。

因此,在计算机中,数据必须有类型,
不同类型的数据在存储单元中所占的字节数不同,
表示的数据范围也不同,数据的运算规则也不同。
在展开讨论C语言的数据类型之前,我们先深入了解
一下整数与浮点数的存储原理。

\subsection{整数的存储与表示}

加法和减法是计算机中最基本的运算,计算机
时时刻刻都离不开它们,所以它们由硬件直接支持。
与此同时,早期计算机电路为了提高加减法的运算效率,
硬件电路需要设计得尽量简单,可是这样一来减法电路
就势必要被阉割掉,因为减法电路的设计要比加法复杂得多。
因此困扰早期计算机设计者的一个重要问题开始浮现:

\vspace{1em}%空出一行距离(当前字体高度)

\indent\textbf{如何用加法电路来实现减法运算?}

\vspace{1em}%空出一行距离(当前字体高度)

要想要实现减法效果,实际上就是在思考怎样才能得到数字0。如果说
a+b=0,那么b就等于-a。此时的b也就是我们想要的减法效果
。显然,从纯数学的角度来说,这个问题是无解的,因为不可能
有两个正数相加等于0。但是不要忘了这里不是纯粹的数学世界,我们发现计算机的
物理电路还有一个重要的特性:\textbf{加法器的自然进位溢出}。
换而言之,我们可以通过一直加,直到数字溢出回到0,从而实现
减法的效果。

\vspace{1em}%空出一行距离(当前字体高度)
\begin{figure}[htbp]
  \centering
  \includegraphics[width=0.4\linewidth]{figures/加法器.png}
  \caption{加法器}
  \label{fig:Adder}
\end{figure}

\newpage


这就好比一个时钟,如\autoref{fig:Clock}所示,时钟的时针此时
指向5点钟,如果我们想让这个时钟归0(也就是指向12点钟)。那么
一共有两种办法,一种是逆时针转动5个小时(这其实就是对应了负数),
另一种是顺时针转动7个小时(而这就是加法溢出)。





\vspace{1em}%空出一行距离(当前字体高度)
\begin{figure}[htbp]
  \centering
  \includegraphics[width=0.4\linewidth]{figures/时钟.png}
  \caption{时钟}
  \label{fig:Clock}
\end{figure}

\indent\textbf{原码反码补码}

\vspace{1em}%空出一行距离(当前字体高度)

了解完以上原理,我们就可以引出整数的存储表示方法了。
在计算机中,整数使用\textbf{补码}来表示。补码的
出现,彻底解决了计算机中加减法运算的问题。从此
计算机不再关注数字的正负,而只需要关注数字的二进制位
即可。这样一来,计算机的加法器就可以直接进行加减法运算。


\begin{enumerate}[label=\ding{\numexpr171+\arabic*}] % 括号编号,使用中文括号编号
  \item 原码:最高位为符号位,0表示正数,1表示负数。其余位表示数值的大小。
  \item 反码:正数的反码与原码相同,负数的反码是将原码中除符号位外,其他位取反。
  \item 补码:正数的补码与原码相同,负数的补码是将反码加1。
\end{enumerate}

\vspace{1em}%空出一行距离(当前字体高度)
\begin{figure}[htbp]
  \centering
  \includegraphics[width=0.4\linewidth]{figures/原码反码补码.jpg}
  \caption{原码、反码与补码的表示}
  \label{fig:complement}

\end{figure}


\newpage




\newpage



C语言的数据类型系统为所有数据(无论变量或常量)提供
统一的分类标准,包括:


\begin{figure}[htbp]
  \centering
  \includegraphics[width=0.8\linewidth]{figures/C语言数据类型.jpeg}
  \caption{C语言数据类型(典型字节数)}
  \label{fig:nature}
\end{figure}










\newpage


