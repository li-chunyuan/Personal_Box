\newpage
\section{数据类型}
\subsection{计算机与数学}


在数学世界里,我们习惯于假定:

\begin{itemize}
  \item 整数可以无限大、无限小;运算可以无限次进行,而不会"装不下"。
  \item 小数可以具有无限精度,如 $\pi$;
  \item 数值的运算是绝对准确的,例如:78与97之和为175,1/3的值是0.33333333……(循环小数)。
\end{itemize}

数学是一门研究抽象问题的学科,数和数的运算都是抽象的。而在计算机中,数据是存放在存储单元中的,
它是具体存在的。而且,存储单元是由有限的字节
构成的,每一个存储单元中
存放数据的范围是有限的,不可能存放“无穷大”的数,
也不能存放循环小数。

所以在真实的计算机系统中,造成这些都不成立的根本原因是\textbf{硬件资源是有限的}:

\begin{itemize}
  \item 内存容量有限,只能分配有限数量的比特来表示一个值;
  \item 寄存器宽度有限,例如常见 CPU 的通用寄存器是 32 位或 64 位;
  \item 加法器、乘法器等算术单元只能处理固定位宽的输入;
\end{itemize}

例如用C程序计算和输出1/3:
程序\texttt{printf("\%f" , 1.0/3.0);}
得到的结果是0.333333,只能得到6位小数,而不是无穷位的小数。

因此,如果想要"无限精度"整数或小数,就只能通过\textbf{软件模拟}来实现(例如 Python 的大整数、GMP 等库),而非依赖硬件原生支持。

从这个角度看,\textbf{数据类型本质上就是在有限资源约束下,对数学世界做出的一个妥协方案}:不同类型对应不同的位宽、不同的范围、不同的精度以及不同的运算规则。数据类型不是单纯用来帮助程序员分类数据的标签,而是\textbf{对内存中比特模式的一种解释方式}。
\newpage

\subsection{数据类型的本质:对比特模式的解释}

计算机的底层存储介质只认识 \texttt{0} 和 \texttt{1}。数据是具体存在的,存放在存储单元中。例如,下面这 4 个字节的二进制内容(为了阅读方便,用空格分组):

\begin{lstlisting}[language={}]
0100 0001 0100 0010 0100 0011 0100 0100
\end{lstlisting}

在不同语境下,它可能表示完全不同的东西:

\begin{table}[H]
  \centering
  \caption{同一比特模式在不同类型解释下的含义示例}\label{tab:datatype}
  \begin{tabular}{ll}
    \toprule
    解释方式 & 含义 \\
    \midrule
    \texttt{char[4]}      & 字符串 \texttt{"ABCD"} \\
    \texttt{int32\_t}     & 一个 32 位有符号整数(某个十进制值) \\
    \texttt{uint32\_t}    & 一个 32 位无符号整数(更大的十进制值) \\
    \texttt{float}        & 一个 IEEE~754 单精度浮点数 \\
    \bottomrule
  \end{tabular}
\end{table}

需要注意的是:\textbf{比特本身并没有变化,变化的只是"如何解释这些比特"的规则},也就是数据类型。

这意味着:

\begin{itemize}
  \item \textbf{没有类型就没有意义}:底层的0和1无法自动告诉计算机"我是哪种数据";
  \item \textbf{编译器必须依赖类型}来决定存储方式、对齐方式,以及在表达式中使用什么样的运算规则。
\end{itemize}

换句话说,类型是 CPU 与编译器之间关于"这些比特该怎么看、怎么算"的协议。因此,在计算机中,数据必须有类型,不同类型的数据在存储单元中所占的字节数不同,表示的数据范围也不同,数据的运算规则也不同。

理解数据类型的本质,是写好 C 语言以及其他底层语言的根基。

\newpage

\subsection{数据类型决定存储格式与运算规则}

数据类型并不是一个“仅供人参考的标签”,它直接决定两件事,存储格式和运算规则。

\subsubsection{存储格式(Representation)}

类型会决定:

\begin{itemize}
  \item 占用的字节数(例如典型平台上 \texttt{int} 为 4 字节、\texttt{char} 为 1 字节);
  \item 在内存中的实际布局,以及是否需要对齐;
  \item 是否有符号(\texttt{signed} vs \texttt{unsigned});
  \item 对于浮点数,如何拆分为符号位、阶码和尾数(IEEE~754)。
\end{itemize}

\subsubsection{运算规则(Operations)}

类型还会决定表达式计算时的具体运算规则:

\begin{itemize}
  \item 使用整数加法器还是浮点运算单元(FPU);
  \item 是按补码规则进行加减,还是作为无符号数做模 $2^n$ 运算;
  \item 运算结果是否可能溢出,溢出之后会发生什么;
  \item 表达式中不同类型之间如何进行\emph{整型提升}和\emph{通常算术转换}。
\end{itemize}

例如下面这个 C 代码片段:

\begin{lstlisting}
char a = 100;
char b = 100;
char c = a + b; 

printf("%c\n", c);      //结果为�
printf("%d",c);         //结果为-56
\end{lstlisting}

在表达式 \texttt{a + b} 中,\texttt{a} 和 \texttt{b} 会
先被\emph{整型提升}为 \texttt{int},实际以 \texttt{int} 进行运算。
但最终赋值给 \texttt{char c} 时,会发生\emph{截断}:只保留结果的低 8 位,
从而可能得到和数学直觉完全不同的值。

因此,\textbf{理解类型对于“值是如何被存储和计算的”至关重要}。

\newpage

\subsection{有限位宽带来的问题:溢出、截断与精度损失}

由于类型对应的位宽是有限的,必然会带来一系列问题,最典型的包括:

\subsubsection{整数溢出(Integer Overflow)}


计算机中的整数类型具有固定的位数,例如:

\begin{table}[h!]
  \centering
  \caption{常见整数类型的位数与表示范围}\label{tab:integer-types}
  \begin{tabular}{ccc}
    \toprule
    \textbf{类型} & \textbf{位数(一般情况)} & \textbf{可表示范围} \\
    \midrule
    \verb|short|        & 16 & $-32768$ $\sim$ $+32767$ \\
    \verb|int|          & 32 & $-2^{31}$ $\sim$ $2^{31}-1$ \\
    \verb|unsigned int| & 32 & 0 $\sim$ $2^{32}-1$ \\
    \verb|long|         & 32/64 & $-2^{31}$ $\sim$ $2^{31}-1$ (或 $-2^{63}$ $\sim$ $2^{63}-1$) \\
    \bottomrule
  \end{tabular}
\end{table}

若表达式的结果超过了该类型所能表示的范围,就会发生 \textbf{整数溢出(Integer Overflow)}。


\noindent\textbf{(1)~无符号整数溢出:定义良好(Well-defined)}\par\vspace{0.5em}


以 \verb|unsigned int| 为例,它只能存储 $0 \sim 4294967295$(即 $0 \sim 2^{32}-1$)。
当执行:

\begin{lstlisting}[language=C]
unsigned int x = 4294967295;
x = x + 1;
\end{lstlisting}

其二进制加法为:
\[
11111111111111111111111111111111_2 + 1 = \textcolor{red}{1}\,00000000000000000000000000000000_2
\]

由于只有 32 位,最高位溢出被丢弃,剩下:

\[
00000000000000000000000000000000_2 = 0
\]

因此:

\[
4294967295 + 1 \equiv 0 \pmod{2^{32}}
\]

也就是说执行下面的代码,得到的结果为0。

\begin{lstlisting}[language=C]
unsigned int x = 4294967295;
printf("%u\n", x + 1);
\end{lstlisting}



\newpage

\noindent\textbf{(2)~有符号整数溢出}\par\vspace{0.5em}

以 \verb|int| 为例,在 32 位系统中其范围为 $-2^{31} \sim 2^{31}-1$(即 $-2147483648 \sim 2147483647$)。

当执行:

\begin{lstlisting}[language=C]
int x = 2147483647;
printf("%d\n", x + 1);
\end{lstlisting}

其二进制补码加法为:
\[
01111111111111111111111111111111_2 + 1 = 10000000000000000000000000000000_2
\]

该值按补码解释为 $-2147483648$。

这也就是为什么上面的代码执行之后的结果为-2147483648


\vspace{0.8em}
\noindent\textbf{(3)~图解整数溢出原理}\par\vspace{0.5em}

如\autoref{fig:integer-overflow} 所示,下图展示了整数溢出的原理。
当无符号整数超过其最大值时,会回绕到0;而有符号整数溢出时,
则会从最大正数翻转为最小负数,这与补码的存储方式直接相关。



\begin{figure}[htbp]
  \centering
  \begin{subfigure}[b]{0.5\textwidth}
    \centering
    \includegraphics[width=\textwidth]{figures/无符号整数溢出.png}
    \caption{无符号整数溢出示意图}
    \label{fig:unsigned-overflow}
  \end{subfigure}
  \hfill
  \begin{subfigure}[b]{0.4\textwidth}
    \centering
    \includegraphics[width=\textwidth]{figures/有符号整数.png}
    \caption{有符号整数溢出示意图}
    \label{fig:signed-overflow}
  \end{subfigure}
  \caption{整数溢出示意图}
  \label{fig:integer-overflow}
\end{figure}

\clearpage






\vspace{0.8em}
\noindent\textbf{(3)~无符号溢出和有符号溢出的区别}\par\vspace{0.5em}




在讨论溢出行为之前,可以先给出一个简单而关键的论断:

\begin{quote}
\textbf{无符号整数的溢出是标准规定的确定行为;有符号整数的溢出是标准未规定的未定义行为。
所以不能完全保证符合预期输出}
\end{quote}

\vspace{1em}

在 C 语言中,有符号整数与无符号整数在发生溢出时具有本质上的区别。
首先,无符号整数(\verb|unsigned|)的溢出属于标准规定的行为。
C 标准明确要求无符号整数采用模 $2^n$ 算术,其中 $n$ 为类型的比特宽度。
也就是说,无符号整数的取值范围始终保持在 $[0,\, 2^n-1]$ 之间,
并在超过最大值时按模 $2^n$ 回绕。
例如,一个 32 位的 \verb|unsigned int| 的最大值为 \verb|0xFFFFFFFF|;
当其再加~1 时,其结果将变为 \verb|0x00000000|。
这种行为在不同的平台、编译器和优化等级下均完全一致,
因此被称为定义良好的行为(well-defined behavior)。

\vspace{1em}

\noindent\textbf{无符号溢出的“环结构”}\par\vspace{0.5em}

以 8 位无符号整数为例:

\[
0 \rightarrow 1 \rightarrow \cdots \rightarrow 254 \rightarrow 255 \rightarrow 0 \rightarrow \cdots
\]

整数在 $0 \sim 255$ 之间构成一个模 256 的环。
\\

与此不同的是,有符号整数(如 \verb|int|、\verb|long| 等)
的溢出属于未定义行为(Undefined Behavior, UB)。所谓未定义行为,
并不是指结果随机,而是 C 标准根本不对溢出之后的结果给出任何保证。
编译器在遇到可能导致有符号整数溢出的代码时有完全的自由:它可以产生某个具体值,
也可以忽略该表达式,甚至可以在优化阶段假定溢出永远不会发生,从而删除某些看似可执行
的代码路径。

出现这种设计的原因,一方面是因为历史上不同处理器使用不同的整数表示方式
(如补码、反码和符号-幅度),导致其溢出行为并不一致;另一方面,将其定义为未定义行为
可以让编译器进行更激进的优化,例如认为 $i+1 > i$ 永远成立。

综上所述,无符号整数的溢出是可预测且具有标准保证的,而有符号整数的溢出则是不受标准约束、具有潜在不可预测性的行为。在实际编程中,应避免任何可能导致有符号整数溢出的情况,否则程序的语义可能会偏离标准规定,并引入难以察觉的错误。

\vspace{1em}
\noindent\textbf{简要比较}\par\vspace{0.5em}

\begin{center}
\begin{tabular}{|c|c|c|c|}
\hline
类型 & 溢出行为 & 标准规定 & 后果 \\ \hline
\verb|unsigned| & 模 $2^n$ 回绕 & 定义良好 & 可预测、可依赖 \\ \hline
\verb|signed| & 无标准规定 & 未定义行为(UB) & 不可预测、不可依赖 \\ \hline
\end{tabular}
\end{center}


\subsubsection{浮点精度损失(Floating-Point Precision Loss)}

浮点数使用有限数量的比特来同时表示范围与精度。很多十进制小数在二进制中是\emph{无法精确表示}的,例如 $0.1$。因此,在实践中你常常会遇到:

\[
0.1 + 0.2 \neq 0.3
\]

这不是计算机“算错了”,而是因为 \texttt{0.1}、\texttt{0.2} 和 \texttt{0.3} 都是以近似值存储,近似值之间的运算不再满足我们在实数域中的直觉。

\subsubsection{截断(Truncation)}

当将一个“位宽较大的类型”赋值给“位宽较小的类型”时,会发生截断:直接丢弃高位,只保留低位。例如:

\begin{lstlisting}
int x = 1000;
char y = x;   // 截断为低 8 位
\end{lstlisting}

若不理解这个过程,很容易在类型转换或结构体内存布局中引入难以察觉的 Bug。

\subsubsection{混合类型与隐式转换}

再看一个常见的“惊喜”示例:

\begin{lstlisting}
unsigned int a = 1;
int b = -2;
printf("%u\n", a + b);
\end{lstlisting}

在这个表达式中,\texttt{a} 是无符号整数、\texttt{b} 是有符号整数。根据 C 语言的\emph{通常算术转换}规则,\texttt{b} 会被转换为无符号类型,然后再参与运算,最终得到一个非常“反直觉”的巨大无符号数。

这些细节都与“数据类型如何被解释和提升”紧密关联。

\subsection{小结}

本节可以归纳为以下几点:

\begin{itemize}
  \item 在计算机里,\textbf{数据类型的本质是对内存中比特模式的一种解释方式};
  \item 类型决定了值的\emph{存储格式}(占多少字节、如何布局、是否有符号)以及\emph{运算规则}(使用哪种算术单元、是否回绕、如何提升和转换);
  \item 硬件资源有限,决定了我们不可能进行“无限精度的数学运算”,从而不可避免地引入溢出、截断和精度损失等问题;
  \item 理解数据类型,是理解后续补码、字节序、整型提升、浮点误差等全部内容的基础。
\end{itemize}

从工程实践的角度看,\textbf{类型不是语法装饰,而是程序正确性和可移植性的基石}。只有真正搞懂“类型到底在替我们做什么”,才能在 C 语言中写出既高效又可靠的代码。



























因此,在计算机中,数据必须有类型,
不同类型的数据在存储单元中所占的字节数不同,
表示的数据范围也不同,数据的运算规则也不同。
在展开讨论C语言的数据类型之前,我们先深入了解
一下整数与浮点数的存储原理。

\subsection{整数的存储与表示}

加法和减法是计算机中最基本的运算,计算机
时时刻刻都离不开它们,所以它们由硬件直接支持。
与此同时,早期计算机电路为了提高加减法的运算效率,
硬件电路需要设计得尽量简单,可是这样一来减法电路
就势必要被阉割掉,因为减法电路的设计要比加法复杂得多。
因此困扰早期计算机设计者的一个重要问题开始浮现:

\vspace{1em}%空出一行距离(当前字体高度)

\indent\textbf{如何用加法电路来实现减法运算?}

\vspace{1em}%空出一行距离(当前字体高度)

要想要实现减法效果,实际上就是在思考怎样才能得到数字0。如果说
a+b=0,那么b就等于-a。此时的b也就是我们想要的减法效果
。显然,从纯数学的角度来说,这个问题是无解的,因为不可能
有两个正数相加等于0。但是不要忘了这里不是纯粹的数学世界,我们发现计算机的
物理电路还有一个重要的特性:\textbf{加法器的自然进位溢出}。
换而言之,我们可以通过一直加,直到数字溢出回到0,从而实现
减法的效果。

\vspace{1em}%空出一行距离(当前字体高度)
\begin{figure}[htbp]
  \centering
  \includegraphics[width=0.4\linewidth]{figures/加法器.png}
  \caption{加法器}
  \label{fig:Adder}
\end{figure}

\newpage


这就好比一个时钟,如\autoref{fig:Clock}所示,时钟的时针此时
指向5点钟,如果我们想让这个时钟归0(也就是指向12点钟)。那么
一共有两种办法,一种是逆时针转动5个小时(这其实就是对应了负数),
另一种是顺时针转动7个小时(而这就是加法溢出)。





\vspace{1em}%空出一行距离(当前字体高度)
\begin{figure}[htbp]
  \centering
  \includegraphics[width=0.4\linewidth]{figures/时钟.png}
  \caption{时钟}
  \label{fig:Clock}
\end{figure}

\indent\textbf{原码反码补码}

\vspace{1em}%空出一行距离(当前字体高度)

了解完以上原理,我们就可以引出整数的存储表示方法了。
在计算机中,整数使用\textbf{补码}来表示。补码的
出现,彻底解决了计算机中加减法运算的问题。从此
计算机不再关注数字的正负,而只需要关注数字的二进制位
即可。这样一来,计算机的加法器就可以直接进行加减法运算。


\begin{enumerate}[label=\ding{\numexpr171+\arabic*}] % 括号编号,使用中文括号编号
  \item 原码:最高位为符号位,0表示正数,1表示负数。其余位表示数值的大小。
  \item 反码:正数的反码与原码相同,负数的反码是将原码中除符号位外,其他位取反。
  \item 补码:正数的补码与原码相同,负数的补码是将反码加1。
\end{enumerate}

\vspace{1em}%空出一行距离(当前字体高度)
\begin{figure}[htbp]
  \centering
  \includegraphics[width=0.4\linewidth]{figures/原码反码补码.jpg}
  \caption{原码、反码与补码的表示}
  \label{fig:complement}

\end{figure}


\newpage




\newpage



C语言的数据类型系统为所有数据(无论变量或常量)提供
统一的分类标准,包括:


\begin{figure}[htbp]
  \centering
  \includegraphics[width=0.8\linewidth]{figures/C语言数据类型.jpeg}
  \caption{C语言数据类型(典型字节数)}
  \label{fig:nature}
\end{figure}










\newpage


