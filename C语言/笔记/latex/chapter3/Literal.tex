\section{从字面量(Literal)到数据类型}

\subsection{何为魔法数字?}%有代码
首先来解释一下什么是字面量。在C语言中,字面量指的是
直接出现在源代码中的常量值,无需标识符即可使用。说得
大白话一点,就是代码中一个又一个数据。如\autoref{lst:literal}
所示,代码中出现的一个又一个常量数字如\texttt{42}、\texttt{'A'}、
\texttt{3.1416}、\texttt{Hi}
就是字面量。


\begin{lstlisting}[language=C,caption={C 语言中字面量的示例},label={lst:literal}]
#include <stdio.h>

int main(void) {
    int x = 42;          // 42 是整型字面量
    char c = 'A';        // 'A' 是字符字面量
    double pi = 3.1416;  // 3.1416 是浮点字面量
    printf("%s\n", "Hi"); // "Hi" 是字符串字面量
    return 0;
}
\end{lstlisting}


在我们日常生活中,我们每天都会接触大量的
数据,比如购物、计数。但我们从来不会纠结于
数据类型或是字面量这些乱七八糟的东西。这是因为
在日常生活中,我们普遍使用\textbf{十进制},
所以对于我们彼此交流的时候,从来都不会纠结于进制、类型问题。


但在计算机中可不是这样,首先在计算机底部存储数字
使用二进制,而其他八进制、十六进制等进制也是常见的表示方式,
再加上由于计算机存储数据的方式不同又诞生了浮点数
、整型数、字符。所以当你冷不丁地给计算机抛出一个数字42时,计算机
并不能理解这个数字到底是什么进制、类型,如\autoref{lst:42}
所示,计算机并不能理解这是十进制的42,还是
八进制的\texttt{34},还是十六进制的\texttt{68}。它更无法理解这到底是
\texttt{int}、\texttt{double}还是\texttt{float}。

计算机此时只能按照默认处理规则进行处理,默认情况下这个\texttt{42}
将被理解为十进制、\texttt{int}类型的数据。但如果你的本意并不是这样,
而是想表达其他进制或类型的数据,那么计算机就会误解你的意思。那么在
工程场景下,往往就会引发严重的错误。


\begin{lstlisting}[language=C,caption={魔法数字实例},label={lst:42}]
                                      42
\end{lstlisting}

计算机遇到这种情况时,
它是无法理解的,因为它不知道这个数字到底是什么进制、类型。
这就好像一个麻瓜突然看到一个魔法师念咒语一样(来自哈利波特),
它根本无法理解魔法师到底在说什么。对于这种情况,我们形象地
称为魔法数字(Magic Number)。



\subsection{定义与概念}
在 \textbf{C 语言标准}(如 ISO/IEC 9899:2018, C17)中,``字面量''(\textit{literal} 或 \textit{literal constant})是一个正式存在的术语。  
它指的是 \textbf{直接出现在源代码中的常量值},无需标识符即可使用。

\begin{center}
\begin{tabular}{lll}
\hline
\textbf{字面量类型} & \textbf{示例} & \textbf{说明} \\
\hline
整型字面量 (integer literal) & \verb|123|, \verb|0x7B|, \verb|010| & 十进制、十六进制、八进制整数 \\
浮点型字面量 (floating-point literal) & \verb|3.14|, \verb|1e-3| & 双精度浮点数 \\
字符字面量 (character constant) & \verb|'A'|, \verb|'\n'| & 单个字符,本质为 \verb|int| 类型 \\
字符串字面量 (string literal) & \verb|"Hello"| & 以 \verb|\0| 结尾的 \verb|char| 数组 \\
\hline
\end{tabular}
\end{center}

\subsection{字面量的具体区分方式}

C 语言里面通过“前缀(\texttt{prefix})”和“后缀(\texttt{suffix})”来对
字面量(\texttt{literal})进行类型和进制的区分。
本节总结了 C 语言中通过前缀 \texttt{(Prefix)} 与
后缀 \texttt{(Suffix)} 来区分字面量的进制与类型的方式。
内容适用于 C89/C99/C11/C23 标准。



\subsubsection{整数:通过前缀区分进制}

整数字面量可使用前缀(\texttt{prefix})确定进制,如 \autoref{tab:int-prefix} 所示。

\begin{table}[h!]
\centering
\caption{整数前缀及其含义}\label{tab:int-prefix}
\begin{tabular}{c|c|c}
\hline
前缀 & 进制 & 示例 \\ \hline
(无前缀) & 十进制 & \verb|123| \\
\verb|0| & 八进制 & \verb|0123| (十进制 83)\\
\verb|0x| / \verb|0X| & 十六进制 & \verb|0x123| (十进制 291)\\
\verb|0b| / \verb|0B|(C23) & 二进制 & \verb|0b1010| (十进制 10)\\
\hline
\end{tabular}
\end{table}


注意:
\begin{itemize}
    \item 以 \texttt{0} 开头的整数默认按八进制解析(如 \texttt{0123} $\neq$ \texttt{123})。
    所以在 C 语言中避免使用前导零,以免引起歧义。这算是一个非常经典的错误原因呢。
    \item  C89/C99/C11 没有二进制前缀, 直到 C23 才正式加入\texttt{0b}作为二进制前缀,
    有些编译器(如 GCC)较早支持。
    \item C 语言中只有十六进制浮点数可以带前缀,其他进制的浮点数一律不允许带前缀。
    (具体见后文)。
\end{itemize}


 
\subsubsection{整数:通过后缀区分类型}



整数字面量的类型可由后缀(Suffix)确定。表 \ref{tab:int-suffix} 汇总了常见后缀。

\begin{table}[h]
\centering
\caption{整数类型后缀}\label{tab:int-suffix}
\begin{tabular}{cccc}
\hline
后缀 & 类型 & 示例 & 中文含义 \\ \hline
(无后缀) & int & \verb|123| & 整型(默认类型) \\

\verb|u| / \verb|U| & unsigned int & \verb|123U| & 无符号整型 \\

\verb|l| / \verb|L| & long int & \verb|123L| & 长整型 \\

\verb|ul| / \verb|UL| & unsigned long & \verb|123UL|, \verb|123LU| & 无符号长整型 \\

\verb|ll| / \verb|LL| & long long int & \verb|123LL| & 长长整型 \\

组合使用 & unsigned long long 等 & \verb|123ULL|, \verb|123LLU| & 无符号长长整型 \\ \hline
\end{tabular}
\end{table}



如\autoref{lst:int-suffix}所示,不同的类型其实意味着\textbf{存储空间}和\textbf{取值范围}
的不同。
所以当你由于不同的应用场景需要表示不同范围的整数时,
可以使用相应的后缀来指定类型。如果这个范围不匹配,
编译器会报错或发出警告。


\begin{lstlisting}[language=C,caption={后缀示例},label={lst:int-suffix}]

123        // int (默认类型)             -2147483648 ~ 2147483647 (32位系统)
123U       // unsigned int              0 ~ 4294967295

123L       // long int                  -2147483648 ~ 2147483647 (32位系统)
123LU      // unsigned long int         0 ~ 4294967295 (32位系统)

123LL      // long long int         -9223372036854775808 ~ 9223372036854775807
123LLU     // unsigned long long int    0 ~ 18446744073709551615

  
//以上给出的数据范围基于常见的32位和64位系统,
//实际范围基于编译器、CPU 架构和操作系统选择的数据模型而异。

\end{lstlisting}

\vspace{1em}
\noindent\textbf{注意}

作为整数字面量后缀:llu 和 ull 完全等价,可以随便使用,没有任何问题。
但在 printf 格式字符串中:必须用 \%llu,不能用 \%ull。所以我推荐大家统一
记忆llu。

\newpage



\subsubsection{浮点数:无前缀(Prefix 不可用)}%有假标题

浮点数\textbf{不能使用前缀} 来表示进制,不能像整数那样写成八进制、二进制形式:

\begin{itemize}
    \item \verb|012.3| (非法,浮点不能为八进制)
    \item \verb|0b1.01| (非法,C 不支持二进制小数)
    \item \verb|0x3.14| (非法的普通十六进制形式)
\end{itemize}



因此:浮点字面量不允许使用任何前缀。但凡事总有例外,在C语言中,十六进制浮点数就能表示
,但其他进制的浮点数一律不允许带前缀。也就是说
\[\textbf{C语言中浮点数只能有两种存在方式,第一种是默认的十进制,第二种是十六进制。}\]

\noindent\textbf{使用前缀表示十六进制浮点常量(C99 及以后)}
\vspace{0.8em}


最开始,浮点字面量\textbf{不能}写成 \verb|012.3|(八进制)、\verb|0b1.01|(二进制)、
\verb|0x3.14|(十六进制)等形式。

但是,从 C99 标准开始,C 语言\textbf{新增}了一类字面量:
\emph{十六进制浮点常量}(hexadecimal floating constant)。  
这一类浮点常量是\textbf{允许}使用 \verb|0x| 或 \verb|0X| 前缀的,
只是语法与整数常量不同,因此很多教材在入门阶段会直接略去不讲。

\vspace{1em}

\noindent\textbf{(1)~十六进制浮点常量的一般形式}\par\vspace{0.5em}

\noindent{十六进制浮点常量的大致结构可以写成:}
\[
\begin{aligned}
  \underbrace{\mathtt{0x}\ \mbox{/}\ \mathtt{0X}}_{\mbox{十六进制前缀}}
  \quad
  \underbrace{\mbox{十六进制数字(可带小数点)}}_{\mbox{significand}}
  \quad
  \underbrace{\mathtt{p}\ \mbox{/}\ \mathtt{P} + \mbox{十进制指数}}_{\mbox{以 2 为底的指数}}
  \quad
  \underbrace{\mbox{浮点后缀 }(\mathtt{f},\mathtt{F},\mathtt{l},\mathtt{L})}_{\mbox{类型说明}}
\end{aligned}
  \]

更口语一点的记忆方式:  
\begin{center}
\verb|0x| \texttt{十六进制小数部分} \verb|p| \texttt{十进制整数指数} \texttt{[可选后缀]}
\end{center}

\noindent
其数值等价于:
\[
  \pm\,\mbox{(十六进制有效数字)} \times 2^{\mbox{指数}}
\]

\newpage

\noindent\textbf{(2)~若干合法示例}\par\vspace{0.5em}

\begin{table}[H]
\centering
\caption{十六进制浮点常量示例及其对应值}\label{tab:hex-float}
\begin{tabular}{ll}
\toprule
字面量 & 对应的十进制值(大致含义) \\
\midrule
\verb|0x1.2p3|       & $(1 + 2/16) \times 2^{3} = 1.125 \times 8 = 9.0$ \\
\verb|0x9A.8p-1|     & $(154 + 8/16) \times 2^{-1} = 154.5 / 2 = 77.25$ \\
\verb|0x1.921fb6p1f| & $\approx 3.1415927\quad(\texttt{float},接近 \pi)$ \\
\bottomrule
\end{tabular}
\end{table}






下面用代码形式再演示一次(需要支持 C99 及以上标准的编译器,例如 \verb|gcc -std=c11|):

\begin{lstlisting}[language=C,caption={十六进制浮点常量示例},label={code:hex-float}]
#include <stdio.h>

int main(void) {
    double b = 0x1.8p1;   // 3.0
    double c = 0x1.fp2;   // 7.75
    float  d = 0x.ap-3f;  // 0.078125f

    printf("b = %f\n", b);
    printf("c = %f\n", c);
    printf("d = %f\n", d);
    return 0;
}
\end{lstlisting}

\vspace{1em}
\noindent\textbf{(3)~为什么 \texttt{0x3.14} 仍然是“非法写法”?}\par\vspace{0.5em}


在上一小节我们给出例子:\verb|0x3.14| “非法的普通十六进制形式”。  
这个结论在 C99 之后\textbf{依然成立},原因是:

\begin{itemize}
  \item 对整数常量来说,形如 \verb|0x3.14| 带小数点,本来就不是合法的“整数十六进制字面量”;
  \item 对十六进制浮点常量来说,\verb|0x3.14| 也\textbf{不完整},因为缺少必须的 \verb|p| 或 \verb|P| 指数部分。
\end{itemize}

也就是说,\verb|0x3.14| 只是一个“十六进制有效数字”,\textbf{不是}完整的“十六进制浮点常量”。  
如果把它补写完整,例如:
\[
  \verb|0x3.14p0|,\quad
  \verb|0x3.14p+0|,\quad
  \verb|0x3.14p4|
\]
这些才是标准允许的、真正合法的十六进制浮点字面量。

\vspace{1em}
\noindent\textbf{(4)~~对“浮点数不能用前缀”说法的精确修正}\par\vspace{0.5em}


因此,可以更严谨地改写原结论:

\begin{itemize}

  \item \textbf{浮点常量}(如 \verb|3.14|、\verb|1e-3|)不能使用 \verb|0|、\verb|0x|、\verb|0b| 等进制前缀;
  \item 从 C99 起,C 语言单独引入了\textbf{十六进制浮点常量},其写法\emph{必须}带有
        \verb|0x|/\verb|0X| 前缀,并以 \verb|p|/\verb|P| 引出二进制指数。
\end{itemize}

在入门阶段,为了避免一次性给出过多语法细节,很多教材会只讲“无前缀的十进制浮点常量”,
并用“浮点数不能用前缀”这样简化的说法帮助初学者形成直觉。  
而在更高阶段学习标准(C99 及以后)时,就需要把十六进制浮点常量这一部分补充进来。

\subsection{浮点数的后缀}%有代码,有表格

浮点字面量只能用后缀确定类型,如表 \ref{tab:float-suffix} 所示。

\begin{table}[h]
\centering
\caption{浮点数字面量类型与后缀}\label{tab:float-suffix}
\begin{tabular}{cccccc}
\hline
后缀 & 类型 & 示例 & 字节数(典型) & 精度(有效数字) & 中文含义 \\ \hline

(无后缀) & double & \verb|3.14| 
& 8 字节
& 约 15–16 位 
& 双精度浮点数(默认) \\

\verb|f| / \verb|F| & float & \verb|3.14f| 
& 4 字节
& 约 6–7 位 
& 单精度浮点数 \\

\verb|l| / \verb|L| & long double & \verb|3.14L| 
& 10–16 字节
& 约 18–33 位 
& 扩展精度浮点数 \\

\hline
\end{tabular}
\end{table}


示例:



\begin{lstlisting}[language=C,caption={浮点数后缀示例},label={code:float-suffix}]
#include <stdio.h>

int main(void) {
3.14f        // float
2.718L       // long double
1e-3         // double(默认类型)
}
\end{lstlisting}


\subsection{总结}

\begin{itemize}
    \item 整数字面量:\textbf{前缀区分进制,后缀区分类型}。
    \item 浮点字面量:\textbf{无前缀,只能用后缀区分类型}。
    \item C99 起提供十六进制浮点字面量(如 \verb|0x1.2p3|)。
\end{itemize}






























