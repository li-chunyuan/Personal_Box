\section{从字面量(Literal)到数据类型}

\subsection{何为魔法数字?}
首先来解释一下什么是字面量。在C语言中,字面量指的是
直接出现在源代码中的常量值,无需标识符即可使用。说得
大白话一点,就是代码中一个又一个数据。如\autoref{lst:literal}
所示,代码中出现的一个又一个常量数字如\texttt{42}、\texttt{'A'}、
\texttt{3.1416}、\texttt{Hi}
就是字面量。


\begin{lstlisting}[language=C,caption={C 语言中字面量的示例},label={lst:literal}]
#include <stdio.h>

int main(void) {
    int x = 42;          // 42 是整型字面量
    char c = 'A';        // 'A' 是字符字面量
    double pi = 3.1416;  // 3.1416 是浮点字面量
    printf("%s\n", "Hi"); // "Hi" 是字符串字面量
    return 0;
}
\end{lstlisting}


在我们日常生活中,我们每天都会接触大量的
数据,比如购物、计数。但我们从来不会纠结于
数据类型或是字面量这些乱七八糟的东西。这是因为
在日常生活中,我们普遍使用\textbf{十进制},
所以对于我们彼此交流的时候,从来都不会纠结于进制、类型问题。


但在计算机中可不是这样,首先在计算机底部存储数字
使用二进制,而其他八进制、十六进制等进制也是常见的表示方式,
再加上由于计算机存储数据的方式不同又诞生了浮点数
、整型数、字符。所以当你冷不丁地给计算机抛出一个数字42时,计算机
并不能理解这个数字到底是什么进制、类型,如\autoref{lst:42}
所示,计算机并不能理解这是十进制的42,还是
八进制的\texttt{34},还是十六进制的\texttt{68}。它更无法理解这到底是
\texttt{int}、\texttt{double}还是\texttt{float}。

计算机此时只能按照默认处理规则进行处理,默认情况下这个\texttt{42}
将被理解为十进制、\texttt{int}类型的数据。但如果你的本意并不是这样,
而是想表达其他进制或类型的数据,那么计算机就会误解你的意思。那么在
工程场景下,往往就会引发严重的错误。


\begin{lstlisting}[language=C,caption={魔法数字实例},label={lst:42}]
                                      42
\end{lstlisting}

计算机遇到这种情况时,
它是无法理解的,因为它不知道这个数字到底是什么进制、类型。
这就好像一个麻瓜突然看到一个魔法师念咒语一样(来自哈利波特),
它根本无法理解魔法师到底在说什么。对于这种情况,我们形象地
称为魔法数字(Magic Number)。



\subsection{定义与概念}
在 \textbf{C 语言标准}(如 ISO/IEC 9899:2018, C17)中,``字面量''(\textit{literal} 或 \textit{literal constant})是一个正式存在的术语。  
它指的是 \textbf{直接出现在源代码中的常量值},无需标识符即可使用。

\begin{center}
\begin{tabular}{lll}
\hline
\textbf{字面量类型} & \textbf{示例} & \textbf{说明} \\
\hline
整型字面量 (integer literal) & \verb|123|, \verb|0x7B|, \verb|010| & 十进制、十六进制、八进制整数 \\
浮点型字面量 (floating-point literal) & \verb|3.14|, \verb|1e-3| & 双精度浮点数 \\
字符字面量 (character constant) & \verb|'A'|, \verb|'\n'| & 单个字符,本质为 \verb|int| 类型 \\
字符串字面量 (string literal) & \verb|"Hello"| & 以 \verb|\0| 结尾的 \verb|char| 数组 \\
\hline
\end{tabular}
\end{center}

\subsection{字面量的具体区分方式}

C 语言里面通过“前缀(\texttt{prefix})”和“后缀(\texttt{suffix})”来对
字面量(\texttt{literal})进行类型和进制的区分。
本节总结了 C 语言中通过前缀 \texttt{(Prefix)} 与
后缀 \texttt{(Suffix)} 来区分字面量的进制与类型的方式。
内容适用于 C89/C99/C11/C23 标准。



\subsubsection{整数:通过前缀区分进制}

整数字面量可使用前缀(\texttt{prefix})确定进制,如 \autoref{tab:int-prefix} 所示。

\begin{table}[h!]
\centering
\caption{整数前缀及其含义}\label{tab:int-prefix}
\begin{tabular}{c|c|c}
\hline
前缀 & 进制 & 示例 \\ \hline
(无前缀) & 十进制 & \verb|123| \\
\verb|0| & 八进制 & \verb|0123| (十进制 83)\\
\verb|0x| / \verb|0X| & 十六进制 & \verb|0x123| (十进制 291)\\
\verb|0b| / \verb|0B|(C23) & 二进制 & \verb|0b1010| (十进制 10)\\
\hline
\end{tabular}
\end{table}


注意:
\begin{itemize}
    \item 以 \texttt{0} 开头的整数默认按八进制解析(如 \texttt{0123} $\neq$ \texttt{123})。
    所以在 C 语言中避免使用前导零,以免引起歧义。这算是一个非常经典的错误原因呢。
    \item  C89/C99/C11 没有二进制前缀, 直到 C23 才正式加入\texttt{0b}作为二进制前缀,
    有些编译器(如 GCC)较早支持。
    \item C 语言中只有十六进制浮点数可以带前缀,其他进制的浮点数一律不允许带前缀。
    (具体见后文)。
\end{itemize}


 
\subsubsection{整数:通过后缀区分类型}



整数字面量的类型可由后缀(Suffix)确定。表 \ref{tab:int-suffix} 汇总了常见后缀。

\begin{table}[h!]
\centering
\caption{整数类型后缀}\label{tab:int-suffix}
\begin{tabular}{c|c|c}
\hline
后缀 & 类型 & 示例 \\ \hline
\verb|u| / \verb|U| & unsigned int & \verb|123U| \\
\verb|l| / \verb|L| & long int & \verb|123L| \\
\verb|ll| / \verb|LL| & long long int & \verb|123LL| \\
\verb|ul| / \verb|UL| & unsigned long etc & \verb|123UL|, \verb|123LU| \\
组合使用 & unsigned long long 等 & \verb|123ULL|, \verb|123LLU| \\
\hline
\end{tabular}
\end{table}

如\autoref{lst:int-suffix}所示,不同的类型其实意味着存储空间和取值范围的不同。
所以当你需要表示更大范围的整数时,可以使用相应的后缀来指定类型。如果这个范围不匹配,
编译器会报错或发出警告。


\begin{lstlisting}[language=C,caption={后缀示例},label={lst:int-suffix}]
123U      // unsigned int             0 ~ 4294967295
123L      // long int              -2147483648 ~ 2147483647 (32位系统)
123UL     // unsigned long int        0 ~ 4294967295 (32位系统) 
123LL     // long long int         -9223372036854775808 ~ 9223372036854775807 
123LLU    // unsigned long long int   0 ~ 18446744073709551615
  
//以上给出的数据范围基于常见的32位和64位系统,
//实际范围基于编译器、CPU 架构和操作系统选择的数据模型而异。

\end{lstlisting}

\vspace{1em}
\noindent\textbf{注意}

作为整数字面量后缀:llu 和 ull 完全等价,可以随便使用,没有任何问题。
但在 printf 格式字符串中:必须用 \%llu,不能用 \%ull。所以我推荐大家统一
记忆llu。





\subsection{浮点数:无前缀(Prefix 不可用)}

浮点数 \textbf{不能使用前缀} 来表示进制,不能像整数那样写成八进制、二进制或普通十六进制形式:

\begin{itemize}
    \item \verb|012.3| (非法,浮点不能为八进制)
    \item \verb|0b1.01| (非法,C 不支持二进制小数)
    \item \verb|0x3.14| (非法的普通十六进制形式)
\end{itemize}

因此:
\[
\textbf{浮点字面量不允许使用任何前缀。}
\]

\subsection{浮点数的后缀(常用)}

浮点字面量只能用后缀确定类型,如表 \ref{tab:float-suffix} 所示。

\begin{table}[h!]
\centering
\caption{浮点字面量后缀}\label{tab:float-suffix}
\begin{tabular}{c|c|c}
\hline
后缀 & 类型 & 示例 \\ \hline
(无后缀) & double & \verb|3.14| \\
\verb|f| / \verb|F| & float & \verb|3.14f| \\
\verb|l| / \verb|L| & long double & \verb|3.14L| \\
\hline
\end{tabular}
\end{table}

示例:

\begin{verbatim}
3.14f        // float
2.718L       // long double
1e-3         // double
\end{verbatim}

\subsection{特殊情况:十六进制浮点(C99 起)}

C99 引入了特殊语法的十六进制浮点常量,格式为:

\[
\verb|0x<hex>.<hex>p<指数>|
\]

其中 \verb|p| 的指数表示 $2^{指数}$。示例:

\begin{verbatim}
double x = 0x1.8p2;   // = (1 + 8/16) * 2^2 = 6.0
\end{verbatim}

注意:
\begin{itemize}
    \item 这是一种“特殊语法”,不是普通意义的“前缀决定进制”。
    \item 仅 C99 及之后标准支持。
\end{itemize}

\subsection{总结}

\begin{itemize}
    \item 整数字面量:\textbf{前缀区分进制,后缀区分类型}。
    \item 浮点字面量:\textbf{无前缀,只能用后缀区分类型}。
    \item C99 起提供十六进制浮点字面量(如 \verb|0x1.2p3|)。
\end{itemize}
































\begin{center}
\begin{tabular}{lllll}
\hline
\textbf{名称} & \textbf{是否存在于语法中} & \textbf{是否有标识符} & \textbf{示例} & \textbf{说明} \\
\hline
字面量 (\textit{literal}) & 是 & 否 & \verb|3.14|, \verb|'A'|, \verb|"abc"| & 直接出现在代码中 \\
常量 (\textit{constant}) & 是 & 是 & \verb|const int a = 10;| & 存储在命名空间中 \\
\hline
\end{tabular}
\end{center}

因此,\verb|10| 是一个字面量,而在 \verb|const int a = 10;| 中的 \verb|a| 是一个常量。

\subsection{延伸说明}
\begin{itemize}
  \item C 语言仅定义了四类基本字面量:整型、浮点型、字符和字符串。
  \item C++ 对字面量进行了扩展,例如用户自定义字面量 \verb|42_km|。
  \item 在 C17 标准第 \S6.4 节中,``literal'' 一词被正式用于描述这些常量。
\end{itemize}



\subsection{总结}
在 C 语言标准中,``字面量''(literal)是正式术语,
表示直接出现在源代码中的常量值,如 \verb|10|、\verb|'A'|、\verb|"abc"| 等。 
它们是语法结构的一部分,而非常量变量。

