\section{C语言的变量与常量}

\subsection{变量与常量的基本定义}

\subsubsection{变量}

变量(Variable)是程序在运行过程中用于存储数据的命名空间。
每个变量在内存中都有一个存储单元,用于保存值,其内容可以在
程序执行过程中改变。你可以把它暂时理解为一个装满数据的容器或者
是一个可以住人的房间。例如:

\begin{lstlisting}[language=C,caption={变量声明示例},label={lst:variables}]
int a = 10;
float b = 3.14;
char c = 'A';
\end{lstlisting}
\vspace{1em}%空出一行距离(当前字体高度)

\subsubsection{常量}
在C语言程序设计中,常量(Constant)是指程序在
运行过程中,其值不可改变的量。合理地定义常量可以提升
代码的可读性和可维护性。
本文对比三种常用常量定义方式:\texttt{\#define} 宏常量、
\texttt{const} 常量
与 \texttt{enum} 枚举常量,并给出实践建议。

\newpage

\begin{enumerate}[label={(\arabic*)}]
\item 符号常量\texttt{\#define} 

\texttt{\#define} 在预处理阶段进行\emph{文本替换},指定用一个
符号名称来代替一个常量。它不参与
类型检查,也不分配存储空间。适合表达编译
期已知且简单的固定值。

\begin{lstlisting}[language=C,caption={使用 \#define 定义宏常量},label={lst:define}]
#include <stdio.h>

#define PI 3.14       //经过指定后,本文件中所有PI都会被替换为3.14。
#define AREA(r) (PI * (r) * (r))      //定义函数宏,用于计算圆面积。

int main() {
    float r = 2.0;
    printf("Area = %.2f\n", AREA(r));
    return 0;
}       //简单理解:只是个文本替换而已。

\end{lstlisting}

在程序编译之前,预处理器会将代码中的
所有 \texttt{PI} 替换为3.14,所有 \texttt{AREA(r)}
 替换为(3.14 * (r) * (r))。

\item 常变量\texttt{const}  
   
\texttt{const}是 C 语言中的一个关键字,
表示“只读(只可读,不可改)”。
它可以用来修饰变量、指针、函数参数等,
使其在程序运行过程中不可被修改。

常变量具有类型信息,占用内存空间,只是其值
不可通过该标识符修改。适合表达接口常量、配置值等。

\begin{lstlisting}[language=C,caption={使用 const 定
  义常量变量},label={lst:const}]
#include <stdio.h>

const int   MAX_CONN = 1024;
const float PI       = 3.14159f;

int main(void) {
    MAX_CONN = 2048; //  编译报错:只读对象不可修改
    printf("%f\n", PI);
    return 0;
}
\end{lstlisting}

\newpage
要注意一件事情:修饰一个变量,只是使其在语义上为“只读”。


也就是说编译器虽然会禁止通过该常变量名字修改,但并
不保证存储区域物理上不可修改。\texttt{const} 变量
仍然占内存,只是不可通过其标识符修改。



\item \texttt{enum} 枚举常量

\texttt{enum} 定义一组自定义的\emph{整数常量},
往往是一组有着共同特性的数据集合或是状态标签。这里
在展开讨论这一常量之前,先讲解一下枚举的概念。

\vspace{1em}%空出一行距离(当前字体高度)

\noindent\textbf{枚举}

枚举(Enumeration)是一种用户自定义的数据
类型,它由一组自定义标签的整数常量组成。每个枚举成员都
对应一个整数值(可以为负数),默认情况下,从 0 开始递增,也可以手动
指定起始值或各个具体值。枚举的主要目的是为了提高代码的可
读性和可维护性。

比如春夏秋冬这四个数据就是一组枚举数据,
它们可以用一个枚举类型来表示;再比如生活中的通讯录,就是
一个非常典型的枚举数据,如\autoref{fig:Contacts} 所示。

\vspace{1em}%空出一行距离(当前字体高度)
\begin{figure}[htbp]
  \centering
  \includegraphics[width=0.8\linewidth]{figures/通讯录.jpeg}
  \caption{生活中的枚举数据:通讯录}
  \label{fig:Contacts}
\end{figure}


电话号码11位数字,按道理来说,一共有 $10^{11}$ 种可能性,
但实际上,我们的通讯录中并不会存入这么多电话号码,
而是只会有其中离散的一部分,比如你亲人
、朋友、同事的号码。此时当我们需要拨打电话时,只能
从通讯录这几个离散的数据里面挑选。

\newpage

枚举的基本规则如\autoref{lst:enum basic}所示


\begin{lstlisting}[language=C,caption={使用 enum 基本规则},label={lst:enum basic}]
enum Color  { RED, GREEN, BLUE };        
// 数值若不自定义,则依次为 0,1,2
enum Weekday{ MON=1, TUE, WED, THU, FRI, SAT, SUN }; 
//若自定义了起始值,则后续依次递增1
enum State { ST_INIT=-5, ST_RUN=8, ST_ERR=9, ST_STOP=100 };
//也可自定义每个枚举值

//注意:枚举数据可以为负数,但必须为整数。
\end{lstlisting}

说完了基本规则,让我们来探讨一下枚举类型变量使用方法,
让我们通过一个示例来加深理解。

\begin{lstlisting}[language=C,caption={使用 enum 定义变量},label={lst:enum var}]
#include <stdio.h>

enum Color { RED, GREEN, BLUE }x;  // 定义枚举类型 Color 并声明变量 x


enum Weekday { MON, TUE, WED, THU, FRI, SAT, SUN };
enum Weekday this_month;  // 声明枚举类型 Weekday 变量 this_month
//这与上面Color的声明方式是等价的。

int main(void) {
    x = RED;  // 给枚举变量 x 赋值  //枚举变量的赋值只能在枚举成员中选择
    this_month = TUE;  // 给枚举变量 this_month 赋值
    printf("Color: %d, Weekday: %d\n", x, this_month);

    return 0;
}
\end{lstlisting}

在上面的示例中,我们定义了两个枚举类型,相信大家已经理解了枚举的
基本用法。接下来我们来接着探讨enum的常量特性。

\texttt{enum}常量是编译期符号,它不同于\texttt{const}常量,也不同于
\texttt{\#define}宏常量在预处理阶段就进行文本替换。
enum常量在编译期被替换为对应的整数值,
不占用任何内存空间,且只能表示整数类型。枚举变量一旦被声明之后
,其值只能在枚举成员中选择,不能赋予其他整数值。这一安全特性使得
枚举常量非常适合表示离散的状态或类别,可以有效提高代码的安全性和可读性。

\end{enumerate}





























\noindent%取消首行缩进
\subsection{命名规则与规范}
\subsubsection{命名规则}
对于字符的组成,变量和常量遵循相同的规则。变量名只能由字母、数字和下划线组成,且首字符不能为
    数字(虽然语法上允许首字符为 \_ ,但在实际编程和标准库约定中有一些潜在风险)
\begin{itemize}
    \item 区分大小写,如 \texttt{age} 与 \texttt{Age} 是不同变量;
    \item 不得与关键字(\texttt{int}, \texttt{return} 等)同名;
\end{itemize}
\subsubsection{代码命名规范}

\begin{enumerate}[label={(\arabic*)}] % 括号编号,使用中文括号编号
  \item 变量命名规范:变量命名通常使用小写字母和下划
  线(例如 \texttt{total\_count}),或者在某些语言中使用驼峰
  式命名。(例如\texttt{totalCount} )
  \item 常量命名规范:
  常量通常采用全大写字母,单词之间用下划线分
  隔。这是为了让开发者在看到常量时,能够立刻
  识别它们是固定值,而不是可变的数据。
  例如:\texttt{MAX\_SIZE}、\texttt{PI}、\texttt{BUFFER\_LIMIT}。  
\end{enumerate}


  这种约定使得代码更加一致,并且提高了可读性和维护性。


\vspace{1em}%空出一行距离(当前字体高度)
\noindent
\subsection{作用域与生命周期}

\subsubsection{变量}

\begin{enumerate}[label={(\arabic*)}] % 括号编号,使用中文括号编号
  \item 局部变量:定义在函数或语句块中,仅在该范围内有效。每次进入函数时该变量都会被
    重新创建并初始化,生命周期仅限于函数调用期间。当函数或语句块执
    行结束后,变量随即被销毁,内存空间被回收;
  \item 全局变量:定义在所有函数外部,全文件范围内可访问。其生命周
    期从程序开始运行到程序结束,在此期间始终占据固定的内存空间。
    全局变量可被同一文件内的所有函数访问(若需跨文件访问,可使用 extern 声明)。
  \item 静态变量(\texttt{static}):使用关键字 static 声明。
它的作用域视其属于局部变量还是全局变量而定,但其生命周期贯穿整个程序执行过程。


也就是说,即使该变量为在函数或语句块中定义的局部变量,其在函数调用结束后也不会被销毁,而是保留其上一次的值;
下次进入函数时会继续使用原来的值,而不是重新初始化。
\end{enumerate}
\clearpage


\vspace{1em}%空出一行距离(当前字体高度)
\noindent%取消首行缩进
\textbf{示例}
\begin{lstlisting}[language=C,caption={局部变量与静态局部变量示例},label={lst:foo_static}]
void foo() {
    int x = 0;           // 局部变量
    static int y = 0;    // 静态局部变量
    x++;
    y++;
    printf("x=%d, y=%d\n", x, y);
}
\end{lstlisting}

\vspace{1em}%空出一行距离(当前字体高度)
\vspace{1em}%空出一行距离(当前字体高度)
\vspace{1em}%空出一行距离(当前字体高度)

\noindent
连续调用 \texttt{foo()} 三次的输出为:

\begin{verbatim}
                              x=1, y=1
                              x=1, y=2
                              x=1, y=3
\end{verbatim}



由于局部变量在函数调用时被创建,并在函数结束后自动销毁,因此每次进
入函数时,x 都会重新初始化为 1。而静态局部变量 y 的生命周期贯穿整个
程序运行过程,因此它的值在每次函数调用后都会保留并继续增加。


\subsubsection{常量}

\begin{enumerate}[label={(\arabic*)}] % 括号编号,使用中文括号编号
  \item 局部常量:当常量在函数内部使用 \texttt{const} 声
  明时,它的作用域仅限于该函数或代码块。这样的常量在函数调用结束后被销毁;
  \item 全局常量:当常量在函数内部使用 const 声
  明时,它的作用域仅限于该函数或代码块。这样的常量在函数调用结束后被销毁。
    全局变量可被同一文件内的所有函数访问(若需跨文件访问,可使用 extern 声明)。
  \item 静态常量(\texttt{static}):与前面静态变量相似,如果常量声明为 static,它
  的生命周期会贯穿整个程序运行期间,即使它的作用域限于声明的文件或函数。
\end{enumerate}






